\documentclass[11pt,letterpaper]{article}
\usepackage{fullpage}
\usepackage[top=2cm, bottom=4.5cm, left=2.5cm, right=2.5cm]{geometry}
\usepackage{amsmath,amsfonts,amssymb}
\usepackage{fancyhdr}
\usepackage{mathrsfs}
\usepackage{xcolor}
\usepackage{hyperref}
\hypersetup{colorlinks=true, linkcolor=blue, linkbordercolor={0 0 1}}

\renewcommand{\arraystretch}{1.75}

\setlength{\parindent}{0.0in}
\setlength{\parskip}{0.05in}

\pagestyle{fancyplain}
\cfoot{\small\thepage}
\headsep 36pt

\begin{document}
\vspace{.2in}
\begin{center}
    {\Large Response to Referee Report \\ JHEP\_252P\_0420\_EDREP032460520}  
\end{center}

\vspace{.25in}

I would like to thank the referee for their detailed and constructive
report, and I hope the changes I have outlined here have addressed their
concerns. Because the report covered several topics, I will address each
of these through the numbering given in the report.
\begin{enumerate}
    \item
    \begin{enumerate}
        \item The issue of the existance of both types of modes being
        present at all times has been expanded upon at the end of section 2 
        following equation (2.19).
        \item If I have misunderstood the referee's comment here, I apologize and would be 
        happy to reconsider their comment. However, I believe the referee may be mistaken on this
        point. Let us allow for integer frequency values for the non-normalizable solutions,
        and examine the general solution for a massless scalar in $d$ dimensions (as 
        per the example put forward in the report). The spatial part of the scalar field
        is given by
        \begin{align}
            \label{e:NN function}
            E_I (x) = K_I \left( \cos x \right)^{\Delta^+} {_2}F_1 \left(\frac{\Delta^+ + \omega_I}{2}, \frac{\Delta^+ - \omega_I}{2}; \frac{d}{2}; \sin^2 x \right)
        \end{align}
        where $K_I$ is a constant, and $\Delta^+ = d$ for a massless field. Now, we
        consider values of $\omega_I = 2i + d$ and examine $E_I$ the origin:
        \begin{align}
            E_I(0) = K_I \; {_2}F_1 \left( i + d, -i; \frac{d}{2}; 0 \right)
        \end{align}
        According to the \href{https://dlmf.nist.gov/15.2}{DLMF}, the hypergeometric function
        ${_2}F_1 (a, b; c; z)$ exists on the disk $| z | < 1$ provided that $c \neq 0, -1, -2, \ldots$
        which the number of dimensions, $d$, always satisfies. Furthermore, we can write
        \begin{align}
            {_2}F_1 (a, b; c; z) = \sum_{s = 0}^\infty \frac{(a)_s (b)_s}{\Gamma(c + s) s!} z^s
        \end{align}
        for all values of $c$. This clearly demonstrates that \eqref{e:NN function} does not
        diverge at the origin and therefore constitutes a valid field decomposition.
        
    \end{enumerate}
\end{enumerate}








\end{document}