\documentclass[11pt,letterpaper]{article}
\usepackage{fullpage}
\usepackage[top=2cm, bottom=4.5cm, left=2.5cm, right=2.5cm]{geometry}
\usepackage{amsmath,amsfonts,amssymb}
\usepackage{fancyhdr}
\usepackage{mathrsfs}
\usepackage{xcolor}
\usepackage{hyperref}
\hypersetup{colorlinks=true, linkcolor=blue, linkbordercolor={0 0 1}}

\renewcommand{\arraystretch}{1.75}

\newcommand{\mc}{\mathcal}

\setlength{\parindent}{0.0in}
\setlength{\parskip}{0.05in}

\pagestyle{fancyplain}
\cfoot{\small\thepage}
\headsep 36pt

\begin{document}
\vspace{.2in}
\begin{center}
    {\Large Response to Referee Report \\ \verb+JHEP_252P_0420_EDREP032460520+}  
\end{center}

\vspace{.25in}

I would like to thank the referee for their detailed and constructive
report, and I hope the changes I have outlined here have addressed their
concerns. Because the report covered several topics, I will address each
of these through the numbering given in the report.
\begin{enumerate}
    \item % 1.
    \begin{enumerate}
        \item The issue of the existence of both types of modes being
        present at all times has been expanded upon at the end of section 2 
        following equation (2.19).
        \item If I have misunderstood the referee's comment here, I apologize and would be 
        happy to reconsider their comment. However, I believe the referee may be mistaken on this
        point. Let us allow for integer frequency values for the non-normalizable solutions,
        and examine the general solution for a massless scalar in $d$ dimensions (as 
        per the example put forward in the report). The spatial part of the scalar field
        is given by
        \begin{align}
            \label{e:NN function}
            E_I (x) = K_I \left( \cos x \right)^{\Delta^+} {_2}F_1 \left(\frac{\Delta^+ + \omega_I}{2}, \frac{\Delta^+ - \omega_I}{2}; \frac{d}{2}; \sin^2 x \right)
        \end{align}
        where $K_I$ is a constant, and $\Delta^+ = d$ for a massless field. Now, we
        consider values of $\omega_I = 2i + d$ and examine $E_I$ at the origin:
        \begin{align}
            E_I(0) = K_I \; {_2}F_1 \left( i + d, -i; \frac{d}{2}; 0 \right)
        \end{align}
        According to the \href{https://dlmf.nist.gov/15.2}{DLMF}, the hypergeometric function
        ${_2}F_1 (a, b; c; z)$ exists on the disk $| z | < 1$ provided that $c \neq 0, -1, -2, \ldots$
        which the number of dimensions, $d$, always satisfies. Furthermore, we can write
        \begin{align}
            {_2}F_1 (a, b; c; z) = \sum_{s = 0}^\infty \frac{(a)_s (b)_s}{\Gamma(c + s) s!} z^s
        \end{align}
        for all values of $c$. This clearly demonstrates that \eqref{e:NN function} does not
        diverge at the origin and therefore constitutes a valid field decomposition.
        \item I have expanded on the issue of including either one or three non-normalizable modes at the end of section 2 and explained that cases where the number of dimensions, the mass, or the driving frequency are tuned to specific values may indeed present resonance channels that are not discussed in the paper. However, the procedure for determining the source terms in the event of such additional resonances is not changed by their presence. 
        
        The particular case raised by the referee can be addressed using additional information from [31]. As addressed under equation (2.6), we can write the general solution for a scalar field with generic frequency $\omega$ as the linear combination of a purely normalizable part $\Phi^+$ with a purely non-normalizable part $\Phi^-$ as ${\phi_1 = C^+ \Phi^+ + C^- \Phi^-}$ where the coefficients are
        \begin{align}
            \label{Cplus}
            C^+ &= \frac{\Gamma(d/2) \Gamma(- \sqrt{d^2 + 4m^2} / 2)}{\Gamma\Big( (\Delta^- + \omega)/2 \Big) \Gamma\Big( (\Delta^- - \omega) / 2 \Big)} \\
            \label{Cminus}
            C^- &= \frac{\Gamma(d/2) \Gamma(\sqrt{d^2 + 4m^2} / 2)}{\Gamma \Big( (\Delta^+ + \omega) / 2 \Big) \Gamma \Big( (\Delta^+ - \omega) / 2 \Big)} \, 
        \end{align}
        and the functions $\Phi^\pm$ are
        \begin{align}
            \label{Phiplus}
            \Phi^+ &= \left( \cos(x) \right)^{\Delta^+} {_2}F_1 \left( \frac{\Delta^+ + \omega}{2}, \frac{\Delta^+ - \omega}{2}, \Delta^+ +1 - \frac{d}{2}; \cos^2 x \right) \\
            \label{Phiminus}
            \Phi^- &= \left( \cos(x) \right)^{\Delta^-} {_2}F_1 \left( \frac{\Delta^- + \omega}{2}, \frac{\Delta^- - \omega}{2}, \Delta^- + 1 - \frac{d}{2} ; \cos^2 x \right)
        \end{align}
        When the frequency $\omega$ is set to an integer that satisfies ${\omega = \omega_i = 2i + \Delta^+}$, \eqref{Cminus} is zero and so $\phi_1 = C^+ \Phi^+$, i.e. the field is \emph{entirely normalizable} for $\omega \geq \Delta^+$. Interestingly, for the particular choice of $m^2 = 0$, $d=4$, and $\omega = 2$ as suggested, it is $C^+$ than vanishes so that $\phi_1 = C^- \Phi^-$ and there is indeed a non-normalizable mode. However, this requires non-generic values of the parameters and so is not explored in detail.

        Finally, I have addressed the case of three non-normalizable modes as an additional example of resonances that require special choices of either mass, number of dimensions, or driving frequency. Such cases are not addressed in this work, as they are not the most broadly applicable. I have also included a caveat below equation (2.19) to caution that not all possible resonances are addressed in this work. However, the procedure for deriving the contributions from special resonances is the same as the procedure for the cases that are included, and so I believe that including such resonances would not produce distinctly new results.
        \item I agree that (4.11) is satisfied for the choice of $d=3$ and $m^2 = -2$. Based on my response above, I have not included this special case; however, I have added a footnote above equation (4.11)  that addresses this case.
    \end{enumerate}
    \item % 2.
    \begin{enumerate}
        \item The final paragraph of section 4 was added to an expanded discussion of the holographic mappings
        between different components of the bulk field to the energy of the boundary CFT in section 2. Additional references to preliminary material on the topic of bulk/boundary mappings that underpin the theoretical motivation for studying non-normalizable modes were also added.
        \item Following the suggestions from the referee's response 2(f) -- resulting in improved discussions of each cases considered in section 4 as they are presented, as well as additional descriptions of the results in section 5 -- I believe the range of parameters and choices of boundary conditions has been made more clear throughout.
        \item The projection of the RHS of (2.14) onto the basis of normalizable modes has been standard since Bizo\'n and Rostworowski in [\href{https://arxiv.org/abs/1104.3702}{1104.3702}] for establishing the first beyond-linear order source term for the scalar field. The examination of the evolution of linearized perturbations of AdS space is covered extensively by Ishibashi and Wald [\href{https://arxiv.org/abs/hep-th/0402184}{hep-th/0402184}]. Since we are considering the scalar field to have both normalizable and non-normalizable modes, however, there are additional considerations. These are the motivation for the discussion at the beginning of section 4 that ultimately lead to the restriction on the mass values.
        \item The details of deriving the flow equations for the amplitudes and phases have been presented extensively in the literature from the point of view of both the Two-Time Formalism [14] and renormalization group method [17]. Since the new physics is encapsulated in the specific nonlinearities of $S_\ell$, I believe that including additional details in how one arrives at (2.17)-(2.18) is best left to the works that are cited. 
        
        The definition of $c^{(3)}_\ell (t)$ follows from the general expression for $c_I^{(3)}(t)$ given below (2.14). I have added a clarifying sentence below (2.16) that links the amplitudes and phases at this order to their first-order counterparts.
        Finally, I have amended the wording about (2.16) to reflect the fact that secular terms are absorbed into the definition of the first-order amplitudes and phases. 
        \item The material concerning resonances from only normalizable modes has been moved to an appendix.
        \item To make the choice of boundary conditions in each section more clear, I have included an explicit form of $\mc F(t)$ for each of the cases explored in this section. Each time, I have clarified the choice of notation for the non-normalizable components. I have also added clarifying statements to the Discussion to remind the reader of the restrictions in place on the values of the mass and choices for the form of the boundary term.
        
        The discussion of static boundary conditions ($\mc F(t) = 0$) for a massive scalar has been moved to an appendix. I believe this will clarify the restriction of equations (3.5), (3.7), and (3.9)-(3.11) to the case of all normalizable modes. I have added an additional comment at the beginning of the appendix to remind the reader of the case being considered.
        \item I have added a clarifying remark below equation (4.16) make it more clear that none of the individual channels vanished when evaluated numerically. Furthermore, because $S_\ell$ changed sign with increasing $\ell$ a check of the sum of the channels was performed and plotted.
        \item The referee again points out an interesting case that does indeed satisfy the resonance condition. However, this again falls into the category of specific tuning of the non-normalizable frequencies that go into the choice of the boundary condition. 
    \end{enumerate}
    \item % 3.
    \begin{enumerate}
        \item References added to other literature that examine TTF in $d = 4$ dimensions.
        \item References have been changed to reflect this correction.
        \item References have been added to the discussion resonances (both vanishing and non-vanishing) for massive scalars with static boundary conditions.
        \item Additional discussion contrasting the results from this work to the numerical investigations in [26] and [28] have been added to section 5.
    \end{enumerate}
    \item % 4.
    \begin{enumerate}
        \item The word ``holographic'' has been removed.
        \item I have changed this sentence to read ``[n]on-normalizable modes are identified through the AdS/CFT correspondence with sources coupling to boundary operators'' to clarify.
        \item The sentence on page 2 as been changed to ``... present numerical evidence ..." to reflect that these resonances have not been shown analytically to always vanish.
        \item The spherical symmetry has been added to the description of the scalar field.
        \item Notation throughout has been changed so that $S_\ell$ always denotes the projection of the third-order source term onto the basis of normalizable eigenfunctions.
    \end{enumerate}
    \item % 5.
    \begin{enumerate}
        \item Please see the response to 1(b) where I have addressed the issue of the regularity of the basis functions $\Phi^\pm_i(x)$.
        \item Since the basis functions for the normalizable modes $e_j(x)$ vanish at the boundary
        ${x \to \pi/2 \; \forall \; j}$, the first term in equation (2.11) is exactly zero while the summation in the second term collapses to a single expression. I believe the wording here accurately describes the simplification of (2.11).
        \item This sentence describes the existence of \emph{naturally vanishing} resonances, such as those given in equations (3.5) and (3.7). These terms evaluate to zero for several choices of the scalar field mass $m \neq 0$. When $m = 0$, [\href{https://arxiv.org/pdf/1407.6273.pdf}{17}] has shown that $\Omega$ and $\Gamma$ vanish for all allowed frequencies by orthogonality of the 
        basis functions. Since chapter 4 demonstrates that there are no resonances that naturally vanish in this fashion, I believe this sentence is correct as it stands.
        \item I have rewritten equation (4.2) as an expansion in $\tilde x$ to make the expression consistent.
        \item I have reduced the three expressions to a single expression that allows for relabeling of the normalizable index and removed the repeated index on the frequencies.
        \item I have corrected $T_\ell$ to read $\overline{T}_\ell$ in the paragraph below (4.2).
        \item I have corrected ``source equations'' to ``source term'' and provided an equation reference.
        \item Yes, (4.9) and (4.10) are indeed applicable for $m^2 = 0$. I have changed the language around (4.9) and (4.10) to make this clearer.
        \item I have changed the caption for figure 5 to make the choice of dimension and mass more clear. The referee is correct that $\overline R^{(-+)}_{i\ell}$ only contributes when $\ell = 0$. However, this is the first value of $n$ that allows for this channel to contribute. While other choices of $n > 4$ would result in additional terms in $\overline R^{(-+)}_{i\ell}$ being non-zero it would also produce more terms in the other channels as well, leading to greater computing requirements. For the purpose of illustration, the $n = 4$ case was chosen to balance these factors.
    \end{enumerate}   
\end{enumerate}

Once again, I would like to thank the referee for their diligent reading of this draft and for their helpful suggestions. I hope I have satisfactorily addressed their concerns and welcome any additional feedback.






\end{document}