\documentclass[11pt,letterpaper]{article}
\usepackage{fullpage}
\usepackage[top=2cm, bottom=4.5cm, left=2.5cm, right=2.5cm]{geometry}
\usepackage{amsmath,amsfonts,amssymb}
\usepackage{fancyhdr}
\usepackage{mathrsfs}
\usepackage{xcolor}
\usepackage{hyperref}
\hypersetup{colorlinks=true, linkcolor=blue, linkbordercolor={0 0 1}}

\renewcommand{\arraystretch}{1.75}

\newcommand{\mc}{\mathcal}

\setlength{\parindent}{0.0in}
\setlength{\parskip}{0.05in}

\pagestyle{fancyplain}
\cfoot{\small\thepage}
\headsep 36pt

\begin{document}
\vspace{.2in}
\begin{center}
    {\Large Response to Referee Report \\ \verb+JHEP_252P_0420_EDREP003650920+}  
\end{center}

\vspace{.25in}

Once again, I wish to thank the referee for their insightful feedback. I have detailed my response to each of the referee's points below.

\begin{enumerate}
    \item % 1.
    \begin{enumerate}
        \item Thank you for this observation. Yes, the expressions for the evolution of the amplitudes and phases do contain contributions from all-normalizable resonances at all times. I have added a comment below (3.7) to reinforce this, as well as addressed it in section 4.
        \item Thank you for noticing this. While some of the terms mentioned had been erroneously included in equation (3.16), this was not clear due to the incorrect amplitude factors. Upon reviewing these terms, I have found additional contributions to the case considered in $\S\!~3.2.1$ and amended equation (3.12) to reflect these changes. Finally, I recalculated the values of $S_\ell$ that are plotted in figures 3 and 4, and included the missing contributions in the flow equations (3.17)-(3.18).
        
        Regarding contributions to (3.20) being inconsistent with the resonance condition $\omega_i + \omega_\gamma = \omega_\beta - \omega_\ell$ when $\omega_\gamma = \omega_\beta$, I agree that these terms could not contribute in this case. However, there are terms that apply when $\omega_i = \omega_\ell$ with $\beta = \gamma + 2\ell + d$. I have calculated the contributions of such terms to (3.20) and recalculated the data that appears in the left hand side of figure 5.
        
        \item I hope I am understanding the comment regarding restrictions on $\bar\omega$ in Appendix C. I have removed the mention of $\bar\omega = \omega_\ell$, since this case is not applicable when non-normalizable modes are present. To further alleviate possible confusion, I have moved the discussion of resonances from all-normalizable modes to after integer values of $\bar\omega$ so that the reader will not have to jump between discussions where non-normalizable modes may or may not be present.  
    \end{enumerate}
    \item % 2
    \begin{enumerate}
        \item I have amended the abstract to limit the masses covered to those within the bounds of $m^2_{BF} < m^2 \leq 0$.
        \item I have included reference [26] at the end of page 5. 
        \item Yes, the ``and'' was intended to be an ``an.'' I have made the appropriate correction.
        \item The duplication has been removed.
        \item I have corrected $T_\ell$ to $\overline{T}_\ell$ above section 3.2.
        \item Indeed, in these two cases $S_\ell = \overline{T}_\ell$.  In later sections, however, we consider cases where $S_\ell$ contains contributions from multiple resonant channels (e.g. Figures 3, 4, 5). In these cases $S_\ell$ is the sum of these channels. Therefore, while the notation may seem redundant for early uses, I believe it provides consistency by always representing the sum of all resonant channels. 
        \item I agree that (2.21) is incorrect. In order to address this -- as well as the comment regarding when $S_\ell$ denotes secular terms, non-secular terms, or both -- I have re-ordered the discussion at the end of section 2.2 to appear before the general expression for $S_\ell$ in (2.2), and I have added a more in-depth explanation of secular terms following the discussion in [17]. Following this, equations (2.22 - 2.23) have been rewritten such that it is more clear that only secular terms from resonant frequencies are included.
    \end{enumerate} 
\end{enumerate}


\end{document}