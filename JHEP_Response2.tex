\documentclass[11pt,letterpaper]{article}
\usepackage{fullpage}
\usepackage[top=2cm, bottom=4.5cm, left=2.5cm, right=2.5cm]{geometry}
\usepackage{amsmath,amsfonts,amssymb}
\usepackage{fancyhdr}
\usepackage{mathrsfs}
\usepackage{xcolor}
\usepackage{hyperref}
\hypersetup{colorlinks=true, linkcolor=blue, linkbordercolor={0 0 1}}

\renewcommand{\arraystretch}{1.75}

\newcommand{\mc}{\mathcal}

\setlength{\parindent}{0.0in}
\setlength{\parskip}{0.05in}

\pagestyle{fancyplain}
\cfoot{\small\thepage}
\headsep 36pt

\begin{document}
\vspace{.2in}
\begin{center}
    {\Large Response to Referee Report \\ \verb+JHEP_252P_0420_EDREP003650920+}  
\end{center}

\vspace{.25in}

Once again, I wish to thank the referee for their insightful feedback. I have detailed my response to each of the referee's points below.

\begin{enumerate}
    \item % 1.
    \begin{enumerate}
        \item 
        \item 
        \item
    \end{enumerate}
    \item % 2
    \begin{enumerate}
        \item I have amended the abstract to limit the masses covered to those within the bounds of $m^2_{BF} < m^2 \leq 0$.
        \item I have included reference [26] at the end of page 5. 
        \item Yes, the ``and'' was intended to be an ``an.'' I have made the appropriate correction.
        \item The duplication has been removed.
        \item I have corrected $T_\ell$ to $\overline{T}_\ell$ above section 3.2.
        \item Indeed, in these two cases $S_\ell = \overline{T}_\ell$.  In later sections, however, we consider cases where $S_\ell$ contains contributions from multiple resonant channels (e.g. Figures 3, 4, 5). In these cases $S_\ell$ is the sum of these channels. Therefore, while the notation may seem redundant for early uses, I believe it provides consistency by always representing the sum of all resonant channels. 
        \item I agree that (2.21) is incorrect. In order to address this -- as well as the comment regarding when $S_\ell$ denotes secular terms, non-secular terms, or both -- I have re-ordered the discussion at the end of section 2.2 to appear before the general expression for $S_\ell$ in (2.2), and I have added a more in-depth explanation of secular terms following the discussion in [17]. Following this, equations (2.22 - 2.23) have been rewritten such that it is more clear that only secular terms from resonant frequencies are included.
    \end{enumerate} 
\end{enumerate}


\end{document}