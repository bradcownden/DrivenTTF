\documentclass[letterpaper,11pt]{article}
\pdfoutput=1 % if you are submitting a pdflatex (i.e. if you have  
% images in pdf, png or jpg format)
\usepackage{jheppub}
\usepackage[utf8]{inputenc} 
%\usepackage[T1]{fontenc} % if needed
%\usepackage[latin1]{inputenc}
\usepackage{graphicx}
\usepackage{amsmath}
\usepackage{amsfonts}
\usepackage{slashed}
\usepackage{amssymb}
\usepackage{hyperref}
\hypersetup{colorlinks=true, linkcolor=blue, citecolor=red, urlcolor=cyan, linktoc=page}
%\usepackage{cite}
\usepackage{xfrac}
\usepackage{empheq}
\usepackage{caption, subcaption}

\newcommand{\p}{\partial}
\newcommand{\oi}{\omega_i}
\newcommand{\oj}{\omega_j}
\newcommand{\ok}{\omega_k}
\newcommand{\ol}{\omega_\ell}
\newcommand{\thi}{\theta_i}
\newcommand{\thj}{\theta_j}
\newcommand{\thk}{\theta_k}
\newcommand{\thl}{\theta_\ell}
\newcommand{\mc}{\mathcal}
\newcommand{\jm}{\ensuremath{j_{max}}}
\newcommand{\ob}{\overline{\omega}}

%%%%%%%%%%%%%%%%%%%%%%%%%%%%%%%%%%%%%%%%%
%%%%%%%%%%%%%%%%%%%%%%%%%%%%%%%%%%%%%%%%%

\title{Arbitrary Dimensions, Massive, Non-normalizable Time-Dependent BCs}

\abstract{}

%%%%%%%%%%%%%%%%%%%%%%%%%%%%%%%%%%%%%%%%%
%%%%%%%%%%%%%%%%%%%%%%%%%%%%%%%%%%%%%%%%%

\begin{document}
\maketitle
\flushbottom
\newpage

%%%%%%%%%%%%%%%%%%%%%%%%%%%%%%%%%%%%%%%%%
%%%%%%%%%%%%%%%%%%%%%%%%%%%%%%%%%%%%%%%%%

\section{Introduction}

%%%%%%%%%%%%%%%%%%%%%%%%%%%%%%%%%%%%%%%%%
%%%%%%%%%%%%%%%%%%%%%%%%%%%%%%%%%%%%%%%%%

\section{Perturbative Expansion}

The backreaction between the metric and the scalar field appears at second order in the perturbation,
\begin{align}
A_2' = - \mu \nu \left[ (\dot \phi_1 )^2 + (\phi_1')^2 + m^2 \phi_1^2 \sec^2 x \right] + \nu' A_2 / \nu
\end{align}
which can be directly integrated to give

\begin{align}
A_2 = -\nu \int^x_0 dy \, \mu \left( (\dot \phi_1 )^2 + (\phi_1')^2 + m^2 \phi_1^2 \sec^2 x \right) \, .
\end{align}
Furthermore, the first non-trivial contribution to the lapse in the boundary time gauge is

\begin{align}
\delta_2 = \int^{\pi/2}_x dy \, \mu \nu \left(  (\dot \phi_1 )^2 + (\phi_1')^2 \right) \, .
\end{align}
For convenience, we have also defined the functions
\begin{align}
\mu (x) = \left( \tan x \right)^{d-1} \quad \text{and} \quad \nu(x) = (d-1) / \mu ' \, .
\end{align}

To aide in evaluating integrals, we first derive the following identities: from the equation for the first-order time-dependent coefficients $c_i$,
\begin{align} 
\ddot c_i + \oi^2 c_i = 0 \quad \Rightarrow \quad \p_t \left(\dot c_i^2 + \oi^2 c_i^2 \right) = \p_t \mathbb C_i = 0 \, ;
\end{align}
from the equation definition of $\hat L$,
\begin{align}
\hat L e_j = -\frac{1}{\mu} \left( \mu e'_j \right)' + m^2 \sec^2 x e_j \quad \Rightarrow \quad \left( \mu e'_j \right)' = \mu \left( m^2 \sec^2 x - \omega_j^2 \right) e_j \, ;
\end{align}
from considering the expression $\left( \mu e'_i e_j \right)'$:
\begin{align}
\left( \mu e'_i e_j \right) ' = \left(m^2 \sec^2 x - \oi^2 \right) \mu e_i e_j + \mu e'_i e'_j \, ;
\end{align}
from permuting $i, j$ above and subtracting to give
\begin{align}
\frac{\left[ \mu (e'_i e_j \oj^2 - e_i e'_j \oi^2 ) \right]'}{(\oj^2 - \oi^2)} = \mu m^2 \sec^2 x e_i e_j + \mu e'_i e'_j \, .
\end{align}

The basis functions $e_j (x)$ are the solutions to the eigenvalue equation
\begin{align}
\hat L e_j(x) = \omega^2_j e_j(x) .
\end{align}
When considering normalizable solutions only, the basis functions become
\begin{align}
e_j(x) &= k_j \left( \cos(x) \right)^{\Delta^+} P_{j}^{(d/2 - 1, \Delta^+ - d/2)} \left( \cos (2x) \right) \\
k_j &= 2 \sqrt{\frac{(j + \Delta^+ /2) \Gamma(j+1) \Gamma(j+\Delta^+)}{\Gamma(j+d/2) \Gamma(j + \Delta^+ - d/2 + 1)}} \, ,
\end{align} 
with eigenvalues $\omega_j = 2j + \Delta^+$, $j \in \mathbb{Z}^*$, and $\Delta^+$ as the positive root of $\Delta ( \Delta - d ) = m^2$. On the other hand, for non-normalizable solutions with arbitrary frequency, the basis functions are
\begin{align}
\label{general basis}
E_\omega (x) =  \left( \cos(x) \right)^{\Delta_+} {_2F_1} \left(\frac{\Delta_+ + \omega}{2}, \frac{\Delta_+ - \omega}{2}, d/2 ; \sin^2 (x) \right) \, .
\end{align}

%%%%%%%%%%%%%%%%%%%%%%%%%%%%%%%%%%%%%%%%%
%%%%%%%%%%%%%%%%%%%%%%%%%%%%%%%%%%%%%%%%%

\section{$\mc O(\epsilon^3)$ Source Terms}
At third order in $\epsilon$, the equation for $\phi_3$ contains a source $S$ given by
\begin{align}
\ddot \phi_3 + \hat L \phi_3 = S = 2 (A_2 - \delta_2) \ddot \phi_1 + (\dot A_2 - \dot \delta_2) \dot\phi_1 + (A_2' -\delta_2' )\phi_1' + m^2 A_2 \phi_1 \sec^2 x
\end{align}
Projecting each of the terms individually onto the eigenbasis $\{ e_\ell \}$:
\begin{align}
\langle \delta_2 \ddot \phi_1, e_\ell \rangle &= - \sum_{i = 0}^\infty \sum_{\substack{j=0 \\ k \neq \ell}}^\infty \sum_{k=0}^\infty \frac{\ok^2 c_k}{\ol^2 - \ok^2} \left[\dot c_i \dot c_j \left(X_{k\ell ij} - X_{\ell k i j} \right) + c_i c_j \left( Y_{ij\ell k} - Y_{ijk\ell} \right) \right] \nonumber \\
& \qquad  - \sum_{i=0}^\infty \sum_{j=0}^\infty \ol^2 c_\ell \left[ \dot c_i \dot c_j P_{ij\ell} + c_i c_j B_{i j \ell} \right] \, , \\
%
\langle A_2 \ddot \phi_1, e_\ell \rangle &= 2 \sum_{i = 0}^\infty \sum_{\substack{j=0 \\ i \neq j}}^\infty \sum_{k=0}^\infty \frac{\ok^2 c_k}{\oj^2 - \oi^2} X_{ijk \ell} \left( \dot c_i \dot c_j + \oj^2 c_i c_j \right) \nonumber \\
& \qquad + \sum_{i = 0}^\infty \sum_{j = 0}^\infty \oj^2 c_j \left( \mathbb C_i P_{j \ell i} + c_i^2 X_{ii j \ell} \right) \, , \\
%
\langle \dot \delta_2 \dot \phi_1 , e_\ell \rangle &= \sum_{i = 0}^\infty \sum_{\substack{j=0 \\ k \neq \ell}}^\infty \sum_{k=0}^\infty \frac{\dot c_k}{\ol^2 - \ok^2} \left[ \p_t \left( \dot c_i \dot c_j \right) \left( X_{k\ell ij} - X_{\ell k i j} \right) + \p_t (c_i c_j) \left(Y_{ij\ell k} - Y_{ijk\ell}\right) \right] \nonumber \\
& \qquad+ \sum_{i=0}^\infty \sum_{j=0}^\infty \dot c_\ell \left[ \p_t \left( \dot c_i \dot c_j \right) P_{ij\ell} + \p_t (c_i c_j) B_{ij\ell} \right] \, , \\
%
\langle \dot A_2 \dot \phi_1, e_\ell \rangle &= -2 \sum_{i=0}^\infty \sum_{j=0}^\infty \sum_{k=0}^\infty  \dot c_k \dot c_j c_i X_{ijk\ell} \, , \\
%
\langle \left( A_2' - \delta_2' \right) \phi_1', e_\ell \rangle &= - 2 \sum_{i = 0}^\infty \sum_{\substack{j=0 \\ i \neq j}}^\infty \sum_{k=0}^\infty \frac{c_k (\dot c_i \dot c_j + \oj^2 c_i c_j)}{\oj^2 -\oi^2} H_{ijk\ell} -m^2 \sum_{i=0}^\infty \sum _{j=0}^\infty \sum_{k=0}^\infty c_i c_j c_k V_{ijk\ell} \nonumber \\
%
& \qquad - \sum_{i=0}^\infty \sum_{j=0}^\infty c_j \left[ c_i^2 H_{iij\ell} + \mathbb C_i M_{j \ell i} \right] \, , \\
%
\langle A_2 \phi_1 \sec^2 x, e_\ell \rangle &= - 2\sum_{i = 0}^\infty \sum_{\substack{j=0 \\ i \neq j}}^\infty \sum_{k=0}^\infty \frac{c_k (\dot c_i \dot c_j + \oj^2 c_i c_j )}{\oj^2 - \oi^2} V_{jki\ell} \nonumber \\
& \qquad - \sum_{i=0}^\infty \sum_{j=0}^\infty c_j \left( c_i^2 V_{jii\ell} + \mathbb C_i Q_{j\ell i} \right) .
\end{align}

Where the forms of X, Y, V, H, B, M, P, and Q are given by
\begin{align}
X_{ijk\ell} &= \int^{\pi/2}_0 dx \, \mu^2 \nu e'_i e_j e_k e_\ell \\
Y_{ijk\ell} &= \int^{\pi/2}_0 dx \, \mu^2 \nu e'_i e'_j e_k e'_\ell \\
V_{ijk\ell} &= \int^{\pi/2}_0 dx \, \mu^2 \nu e_i e_j e'_k e_\ell \sec^2 x \\
H_{ijk\ell} &= \int^{\pi/2}_0 dx \, \mu^2 \nu' e'_i e_j e'_k e_\ell \\
B_{ij\ell} &= \int^{\pi/2}_0 dx \, \mu \nu e'_i e'_j \int^x_0 dy \, \mu e^2_\ell \\
M_{ij\ell} &= \int^{\pi/2}_0 dx \, \mu \nu' e'_i e_j \int^x_o dy \, \mu e_\ell^2 \\
P_{ij\ell} &= \int^{\pi/2}_0 dx \, \mu \nu e_i e_j \int^x_0 dy \, \mu e^2_\ell \\
Q_{ij\ell} &= \int^{\pi/2}_0 dx \, \mu \nu e_i e_j \sec^2 x \int^x_0 dy \, \mu e^2_\ell
\end{align}

Collecting terms together gives the expression for $S_\ell = \langle S, e_\ell \rangle$:
\begin{align}
S_\ell &= \sum_{\substack{i, j, k \\ k \neq \ell}}^\infty \frac{1}{\ol^2 - \ok^2} \Big[ F_k(\dot c_i \dot c_j) \left(X_{k\ell i j} - X_{\ell k i j} \right) + F_k(c_i c_j) \left(Y_{ij\ell k} - Y_{ijk\ell} \right) \Big] \nonumber \\
%
& \quad +2 \sum_{\substack{i,j,k \\ i \neq j}}^\infty \frac{c_k D_{ij}}{\oj^2 - \oi^2} \Big[  2\ok^2 X_{ijk\ell} - H_{ijk\ell} -m^2 V_{jki\ell} \Big] - \sum_{i,j,k}^\infty c_i \Big[ 2 \dot c_j \dot c_k X_{ijk\ell} + m^2 c_j c_k V_{ijk\ell} \Big] \nonumber \\ 
%
& \quad + \sum_{i,j}^\infty \Big[ F_\ell (\dot c_i \dot c_j) P_{ij\ell} + F_\ell (c_i c_j) B_{ij\ell} + 2\oj^2 c_j \left( c_i^2 X_{iij\ell} + \mathbb C_i P_{j\ell i} \right) \nonumber \\
%
& \qquad - c_j \left( c^2_i (H_{iij\ell} + m^2 V_{jii\ell} ) + \mathbb C_i (M_{j\ell i} + m^2 Q_{j\ell i}) \right) \Big] \, ,
\end{align}
where $F_k(z) = \dot c_k \dot z - 2\ok^2 c_k z$, $D_{ij} = \dot c_i \dot c_j + \omega^2_j c_i c_j$, and $\mathbb C_i = \dot c_i^2 + \oi^2 c_i^2$.

Using the solution $c_i(t) = a_i \cos (\oi t + b_i) = a_i \cos \theta_i$, the source term becomes

\begin{align}
\label{general source}
S_\ell &=\frac{1}{4} \sum_{\substack{i,j,k \\ k \neq \ell}}^\infty \frac{a_i a_j a_k \ok}{\ol^2 - \ok^2} \Big[ Z^-_{ijk\ell} (\oi + \oj - 2\ok) \cos (\thi + \thj - \thk) - Z^-_{ijk\ell} (\oi + \oj + 2\ok) \cos (\thi + \thj + \thk) - \nonumber \\
%
&\qquad + Z^+_{ijk\ell} (\oi - \oj + 2\ok)  \cos(\thi - \thj + \thk) - Z^+_{ijk\ell} (\oi - \oj - 2\ok) \cos (\thi - \thj - \thk) \Big] \nonumber \\
%
& + \frac{1}{2}\sum_{\substack{i,j,k \\ i \neq j}}^\infty a_i a_j a_k \oj \left( H_{ijk\ell} + m^2 V_{jki\ell} - 2\ok^2 X_{ijk\ell} \right) \Big[ \frac{1}{\oi - \oj} \left( \cos (\thi - \thj - \thk)  + \cos(\thi - \thj + \thk) \right) \nonumber \\
%
& \qquad - \frac{1}{\oi + \oj} \left( \cos (\thi + \thj - \thk)  + \cos ( \thi + \thj + \thk) \right) \Big] \nonumber \\
%
& - \frac{1}{4} \sum_{i,j,k}^\infty a_i a_j a_k \Big[ \left( 2\oj \ok X_{ijk\ell} + m^2 V_{ijk\ell} \right)\cos(\thi + \thj - \thk) -  \left( 2\oj\ok X_{ijk\ell} - m^2 V_{ijk\ell} \right) \cos(\thi - \thj - \thk) \nonumber \\
%
& \qquad + \left(2\oj \ok X_{ijk\ell} + m^2 V_{ijk\ell} \right) \cos (\thi - \thj + \thk) - \left( 2\oj\ok X_{ijk\ell} - m^2 V_{ijk\ell} \right) \cos(\thi + \thj + \thk) \Big] \nonumber \\
%
& + \frac{1}{4} \sum_{i,j}^\infty a_i a_j a_\ell \ol \Big[ \tilde Z^-_{ij\ell} (\oi + \oj - 2\ol) \cos (\thi + \thj - \thl) - \tilde Z^-_{ij\ell} (\oi + \oj + 2\ol) \cos(\thi + \thj +  \thl) \nonumber \\
%
& \qquad + \tilde Z^+_{ij\ell} (\oi - \oj + 2\ol) \cos(\thi - \thj + \thl)  - \tilde Z^+_{ij\ell} (\oi - \oj - 2\ol) \cos( \thi - \thj - \thl)  \Big] \nonumber \\
%
& - \frac{1}{4} \sum_{i,j}^\infty a_i^2 a_j \left( H_{iij\ell} + m^2 V_{jii\ell} - 2\oj^2 X_{iij\ell} \right) \big[ \cos (2\thi - \thj) + \cos (2\thi + \thj) \big] \nonumber \\
%
& - \frac{1}{2} \sum_{i,j}^\infty a_i^2 a_j \left( H_{iij\ell} + m^2 V_{jii\ell} - 2\oj^2 X_{iij\ell} + 4\oi^2 \oj^2 P_{j\ell i} + 2\oi^2 (M_{j\ell i} + m^2 Q_{j\ell i}) \right) \cos \thj . \hspace{-0.2in}
\end{align} 
To simplify the above expression, we have defined
\begin{align}
Z^{\pm}_{ijk\ell} = \oi \oj \left( X_{k\ell ij} - X_{\ell kij} \right) \pm \left( Y_{ij\ell k} - Y_{ijk\ell} \right) \quad \text{and} \quad \tilde Z^{\pm}_{ij\ell} = \oi \oj P_{ij\ell} \pm B_{ij\ell} \, .
\end{align}

Using integration by parts to remove the derivative from $\nu$ in the definitions of $H_{ijk\ell}$ and $M_{ij\ell}$, we can show that
\begin{align}
H_{ijk\ell} &= \oi^2 X_{kij\ell} + \ok^2 X_{ijk\ell} - Y_{ij\ell k}  - Y_{\ell kji}   - m^2 V_{kji\ell} -m^2 V_{ijk\ell} \\
M_{ij\ell} &= \oi^2 P_{ij\ell} - B_{ij\ell} -m^2 Q_{ij\ell}
\end{align}

%%%%%%%%%%%%%%%%%%%%%%%%%%%%%%%%%%%%%%%%%
%%%%%%%%%%%%%%%%%%%%%%%%%%%%%%%%%%%%%%%%%

\section{Resonances From Normalizable Solutions}
\label{sec: norm res}
Consider the case where each of the basis functions are given by normalizable solutions. After time-averaging, resonant contributions come from the set of conditions
\begin{align}
\label{gen res}
\oi \pm \oj \pm \ok = \pm \ol \,
\end{align}
which separates into three distinct cases
\begin{align}
\oi + \oj + \ok &= \ol \qquad (+++) \\
\oi - \oj - \ok &= \ol \qquad (+--) \\
\oi + \oj - \ok &= \ol \qquad (++-)
\end{align}

%%%%%%%%%%%%%%%%%%%%%%%%%%%%%%%%%%%%%%%%%

\subsection{$(+++)$}

These resonant contributions come from the condition $\oi + \oj + \ok = \ol$, and are of the form
\begin{align}
S_\ell = \underbrace{\sum_{i=0}^\infty \sum_{j=0}^\infty \sum_{k=0}^\infty}_{\oi + \oj + \ok = \, \ol} \Omega_{ijk\ell} \, a_i a_j a_k \cos \left( \thi + \thj + \thk \right) + \ldots \, ,
\end{align}
where
\begin{align}
\label{omega}
\Omega_{ijk\ell} &= -\frac{1}{12}H_{ijk\ell} \frac{\oj (\oi + \ok +2\oj)}{(\oi + \oj)(\oj + \ok)} - \frac{1}{12} H_{ikj\ell} \frac{\ok (\oi + \oj + 2\ok)}{(\oi + \ok)(\oj + \ok)}- \frac{1}{12} H_{jik\ell} \frac{\oi (\oj + \ok +2\oi)}{(\oi + \oj)(\oi + \ok)} \nonumber \\
%
& \quad - \frac{m^2}{12} V_{ijk\ell} \left( 1 + \frac{\oj}{\oj + \ok} + \frac{\oi}{\oi + \ok} \right) - \frac{m^2}{12} V_{jki\ell} \left( 1 + \frac{\oj}{\oi + \oj} + \frac{\ok}{\oi + \ok} \right) \nonumber \\
%
& \quad - \frac{m^2}{12} V_{kij\ell} \left( 1 + \frac{\oi}{\oi + \oj} + \frac{\ok}{\oj + \ok} \right)  + \frac{1}{6} \oj \ok X_{ijk\ell} \left( 1 + \frac{\oj}{\oi + \ok} + \frac{\ok}{\oi + \oj} \right) \nonumber \\
%
& \quad + \frac{1}{6} \oi \ok X_{jki\ell} \left( 1 + \frac{\oi}{\oj + \ok} + \frac{\ok}{\oi + \oj} \right) + \frac{1}{6} \oi \oj X_{kij\ell} \left( 1 + \frac{\oi}{\oj + \ok} + \frac{\oj}{\oi + \ok} \right) \nonumber \\
%
& \quad - \frac{1}{12} Z^-_{ijk\ell} \left( \frac{\ok}{\oi + \oj} \right) - \frac{1}{12} Z^-_{ikj\ell} \left( \frac{\oj}{\oi + \ok} \right) - \frac{1}{12} Z^-_{jki\ell}  \left( \frac{\oi}{\oj + \ok} \right) \, .
\end{align}

%%%%%%%%%%%%%%%%%%%%%%%%%%%%%%%%%%%%%%%%%

\subsection{$(+--)$}

These contributions arise from the condition $\oi - \oj - \ok = \ol$, are of the form
\begin{align}
S_\ell = \sum_{j=0}^\infty \sum_{k=0}^\infty \Gamma_{(j+k+\ell) jk\ell} \, a_j a_k a_{(j+k+\ell)} \cos \left( \theta_{j+k+\ell} - \thj - \thk \right) + \ldots \, ,
\end{align}
where
\begin{align}
\label{gamma}
\Gamma_{ijk\ell} &= \frac{1}{4} H_{ijk\ell} \frac{\oj (\ok - \oi + 2\oj)}{(\oi - \oj)(\oj + \ok)} + \frac{1}{4} H_{jki\ell} \frac{\ok (\oj - \oi + 2\ok)}{(\oi - \ok)(\oj + \ok)} + \frac{1}{4} H_{kij\ell} \frac{\oi (\oj + \ok - 2\oi)}{(\oi - \oj)(\oi - \ok)} \nonumber \\
% 
& \quad -\frac{1}{2} \oj \ok X_{ijk\ell} \left( \frac{\ok}{\oi - \oj} + \frac{\oj}{\oi - \ok} - 1\right) + \frac{1}{2} \oi \ok X_{jki\ell} \left( \frac{\ok}{\oi - \oj} + \frac{\oi}{\oj + \ok} - 1 \right) \nonumber \\
%
& \quad + \frac{1}{2} \oi \oj X_{kij\ell} \left( \frac{\oj}{\oi - \ok} + \frac{\oi}{\oj + \ok} -1 \right) + \frac{m^2}{4} V_{jki\ell} \left( \frac{\oj}{\oi - \oj} + \frac{\ok}{\oi - \ok} -1\right) \nonumber \\
%
& \quad - \frac{m^2}{4} V_{kij\ell} \left( \frac{\oi}{\oi - \oj} + \frac{\ok}{\oj + \ok} + 1\right) - \frac{m^2}{4} V_{ijk\ell} \left( \frac{\oi}{\oi - \ok} + \frac{\oj}{\oj + \ok} + 1 \right) \nonumber \\
%
& \quad + \frac{1}{4} Z^-_{kji\ell} \left( \frac{\oi}{\oj + \ok}\right) - \frac{1}{4} Z^+_{ijk\ell} \left( \frac{\ok}{\oi - \oj} \right) - \frac{1}{4} Z^+_{jki\ell} \left( \frac{\oj}{\oi - \ok}\right) \, .
\end{align}

%%%%%%%%%%%%%%%%%%%%%%%%%%%%%%%%%%%%%%%%%

\subsection{Naturally Vanishing Resonances}

It has been shown that when $m=0$, and only normalizable modes are considered, \eqref{omega} and \eqref{gamma} vanish by the orthogonality of the basis functions. {\bf Maybe show that mass-dependent terms vanish for normalizable modes?}

%%%%%%%%%%%%%%%%%%%%%%%%%%%%%%%%%%%%%%%%%

\subsection{$(++-)$}
\label{subs: ttf resonances}

These contributions arise from the resonant condition $\oi + \oj = \ok + \ol$, can be written as
\begin{align}
S_\ell &= T_\ell a^3_\ell \cos (\thl + \thl - \thl) + \sum_{i \neq \ell}^\infty R_{i \ell} \, a^2_i a_\ell \cos(\thi + \thl - \thi) \nonumber \\
& \qquad + \sum_{i \neq \ell}^\infty \sum_{j \neq \ell}^\infty S_{i j (i + j - \ell) \ell} \, a_i a_j a_{(i + j - \ell)} \cos(\thi + \thj - \theta_{i + j -\ell} ) + \ldots
\end{align}
where each of the coefficients is given by
\begin{align}
S_{ijk\ell} &= - \frac{1}{4} H_{kij\ell} \frac{\oi (\oj - \ok + 2\oi)}{(\oi - \ok)(\oi + \oj)} -\frac{1}{4} H_{ijk\ell} \frac{\oj (\oi - \ok + 2\oj)}{(\oj - \ok)(\oi + \oj)} - \frac{1}{4} H_{jki\ell} \frac{\ok ( \oi + \oj - 2\ok)}{(\oi - \ok)(\oj - \ok)} \nonumber \\
%
& \quad - \frac{1}{2} \oj \ok X_{ijk\ell} \left( \frac{\oj}{\oi - \ok} - \frac{\ok}{\oi + \oj} + 1 \right) - \frac{1}{2} \oi \ok X_{jki\ell} \left( \frac{\oi}{\oj - \ok} - \frac{\ok}{\oi + \oj} + 1 \right) \nonumber \\
%
& \quad + \frac{1}{2} \oi \oj X_{kij\ell} \left( \frac{\oi}{\oj - \ok} + \frac{\oj}{\oi - \ok} + 1 \right) - \frac{m^2}{4} V_{ijk\ell} \left( \frac{\oi}{\oi - \ok} + \frac{\oj}{\oj - \ok} + 1\right) \nonumber \\
%
& \quad + \frac{m^2}{4} V_{jki\ell} \left( \frac{\ok}{\oi - \ok} - \frac{\oj}{\oi + \oj} - 1 \right) + \frac{m^2}{4} V_{kij\ell} \left( \frac{\ok}{\oj - \ok} - \frac{\oi}{\oi + \oj} - 1 \right) \nonumber \\
%
& \quad + \frac{1}{4}  Z^-_{ijk\ell} \left( \frac{\ok}{\oi + \oj}\right)  + \frac{1}{4}  Z^+_{ikj\ell} \left( \frac{\oj}{\oi - \ok}\right) + \frac{1}{4} Z^+_{jki\ell} \left( \frac{\oi}{\oj - \ok} \right) \, ,
\end{align}
\begin{align}
R_{i\ell} &= \left(\frac{\oi^2}{\ol^2 - \oi^2} \right) \big( Y_{i\ell \ell i} - Y_{i\ell i \ell} + \ol^2 ( X_{i\ell i \ell} - X_{\ell i \ell i}) \big) + \left(\frac{\oi^2}{\ol^2 - \oi^2}\right) \big( H_{\ell i i\ell} + m^2 V_{ii\ell \ell} - 2\oi^2 X_{\ell i i \ell} \big) \nonumber \\
%
& - \left(\frac{\ol^2}{\ol^2 - \oi^2} \right) \big( H_{i\ell i \ell} + m^2 V_{\ell i i \ell} - 2\oi^2 X_{i\ell i\ell} \big) - \frac{m^2}{4}(V_{i\ell i \ell} + V_{ii\ell \ell} ) + \oi^2 \ol^2 (P_{ii\ell} - 2P_{\ell \ell i}) \nonumber \\
%
& - \oi\ol X_{i\ell i \ell} - \frac{3m^2}{2} V_{\ell ii \ell} - \frac{1}{2} H_{ii\ell \ell} + \ol^2 B_{ii\ell} - \oi^2 M_{\ell \ell i} - m^2 \oi^2 Q_{\ell \ell i} \, ,
\end{align}
\begin{align}
T_{\ell} &= \frac{1}{2} \ol^2 \left( X_{\ell \ell \ell \ell} + 4 B_{\ell \ell \ell} -2 M_{\ell \ell \ell} - 2m^2 Q_{\ell \ell \ell} \right) -\frac{3}{4} \left( H_{\ell \ell \ell \ell} + 3m^2 V_{\ell \ell \ell \ell} \right) \, . \hspace{1.0in}
\end{align}

%%%%%%%%%%%%%%%%%%%%%%%%%%%%%%%%%%%%%%%%%
%%%%%%%%%%%%%%%%%%%%%%%%%%%%%%%%%%%%%%%%%

\section{Resonances From Non-normalizable Modes}

We now consider the case when at least one of the $e_i(x), e_j(x), e_k(x)$ is a non-normalizable mode. Since the boundary condition has been set to be a single non-normalizable mode, any non-normalizable modes in the source term must exactly cancel; therefore, at least two of the modes must be non-normalizable. This assumption breaks some of the symmetries that contributed to the previous expressions for resonance channels, and so the resonance conditions must be re-examined starting from the source expression \eqref{general source}.

\subsection{Two General, Non-normalizable Modes}

As a first case, let us assume that the two non-normalizable modes have constant, generic (i.e., non integer) frequency values, $\ob$. Applying the time-averaging procedure to the source $S_\ell$ once again eliminates all contributions except those that satisfy \eqref{gen res}. Since the basis onto which we are projecting is normalizable, we know that $\omega_\ell$ is given by $\omega_\ell = 2\ell + \Delta^{+}$. We are now free to choose any one of $\{\omega_i, \omega_j, \ok\}$ to be normalizable and consider when the resonance condition is satisfied. In particular, we find that the following combinations are resonant:
\begin{align}
\oi - \oj + \ok - \ol &= 0 \qquad \Rightarrow \qquad \text{either $\oi$ or $\ok$ is normalizable} \\
\oi + \oj - \ok - \ol &= 0 \qquad \Rightarrow \qquad \text{either $\oi$ or $\oj$ is normalizable} \\
\oi - \oj - \ok + \ol &= 0 \qquad \Rightarrow \qquad \text{either $\oj$ or $\ok$ is normalizable.}
\end{align}
When any of these resonance conditions is met, the remaining normalizable mode will have a frequency equal to $\ol$, collapsing all sums over frequencies so that
\begin{align}
S_\ell = \overline{T}_\ell \, a_\ell^3 \cos (\thl) \, .
\end{align}
Collecting the appropriate terms in \eqref{general source}, and evaluating the each possible resonance (being careful not to violate restrictions placed on the sums), we find that
\begin{align}
\overline{T}_\ell &= \frac{3 \ol^2 \ob^2}{\ol^2 - \ob^2} X_{\ob \ob \ell \ell} - \frac{\ob^2 (\ol^2 + \ob^2)}{\ol^2 -\ob^2} X_{\ell \ell \ob \ob} + \ol^2 X_{\ob \ob \ell \ell} - \ob^2 X_{\ell \ell \ob \ob} + \frac{2 \ol^2}{\ol^2 - \ob^2} Y_{\ob \ob \ell \ell} \nonumber \\
\quad & - \frac{2 \ob^2}{\ol^2 - \ob^2} Y_{\ell \ell \ob \ob} - \frac{1}{2} H_{\ell \ob \ob \ell} + 2m^2 \left( \frac{4 \ol^2 - 3 \ob^2}{\ol^2 - \ob^2}  \right) V_{\ell \ell \ob \ob}  - \frac{m^2}{2} V_{\ob \ob \ell \ell} \nonumber \\
\quad & + \ol^2 \ob^2 P_{\ob \ob \ell} - 3 \ob^2 \ol^2 P_{\ell \ell \ob} + \ol^2 B_{\ob \ob \ell} + \ob^2 B_{\ell \ell \ob} \, ,
\end{align}
after expanding out $Z^{\pm}_{ijk\ell}$, $H_{ijk\ell}$, $\tilde{Z}^{\pm}_{\ij\ell}$, and $M_{ij\ell}$.



\subsection{Special Values of Non-normalizable Frequencies}

\subsubsection{Add to an integer}

\section{Examining a Special Case For $m^2$}

Returning to the solutions of the eigenvalue equation $\hat L e_j(x) = \omega^2_j e_j(x)$, it has so far been implicit that the the value of $\sqrt{d^2 + 4m^2}$ is taken to be both non-integer and greater than two. However, the Breitenholmer-Freeman bound allows for the existence of negative mass-squared, i.e. tachyonic, scalar fields in AdS$_{d+1}$, provided that $d^2 + 4m^2 \geq 0$. Let us now consider a class of solutions that include a tachyonic scalar: when $\sqrt{d^2 + 4m^2} \in \mathbb{Z}^*$. 

The general solution in this case can be written as the sum \cite{hep-th/9805171}
\begin{align}
\Psi(x) = C^{(+)} \Phi^{(+)} + C^{(-)} \Phi^{(-)}
\end{align}
where
\begin{align}
\Phi^{(+)} &= \left( \cos (x) \right)^{\Delta^+} {_2 F_1}\left( \frac{\Delta^+ + \omega}{2}, \frac{\Delta^+ - \omega}{2}, 1 + \frac{1}{2} \sqrt{d^2 + 4m^2} \, ; \cos^2 x \right) \, ,
\end{align}
and the form of $\Phi^{(-)}$ depends on the value of $\sqrt{d^2 + 4m^2}$. When $4m^2 = -d^2$, we have
\begin{align}
\Phi^{(-)} &= \text{equation (43) of hep-th/9805171}
\end{align}
while for $\sqrt{d^2 + 4m^2} \in \mathbb{Z}^+$,
\begin{align}
\Phi^{(-)} &= \text{equation (44) of the same.}
\end{align}

\subsection{Boundary Condition is a Superposition/Fourier Integral of Non-normalizable Modes}

\acknowledgments

This research was enabled in part by support provided by WestGrid (\href{www.westgrid.ca}{www.westgrid.ca}) and Compute Canada (\href{www.computecanada.ca}{www.computecanada.ca}).

%%%%%%%%%%%%%%%%%%%%%%%%%%%%%%%%%%%%%%%%%
%%%%%%%%%%%%%%%%%%%%%%%%%%%%%%%%%%%%%%%%%

\bibliographystyle{JHEP}
%\bibliography{}

%%%%%%%%%%%%%%%%%%%%%%%%%%%%%%%%%%%%%%%%%
%%%%%%%%%%%%%%%%%%%%%%%%%%%%%%%%%%%%%%%%%

\end{document}

%%%%%%%%%%%%%%%%%%%%%%%%%%%%%%%%%%%%%%%%%
%%%%%%%%%%%%%%%%%%%%%%%%%%%%%%%%%%%%%%%%%


