\documentclass[letterpaper,11pt]{article}
\pdfoutput=1 % if you are submitting a pdflatex (i.e. if you have  
% images in pdf, png or jpg format)
\usepackage{jheppub}
\usepackage[utf8]{inputenc} 
%\usepackage[T1]{fontenc} % if needed
%\usepackage[latin1]{inputenc}
\usepackage{graphicx}
\usepackage{amsmath}
\usepackage{amsfonts}
\usepackage{slashed}
\usepackage{amssymb}
\usepackage{hyperref}
\hypersetup{colorlinks=true, linkcolor=blue, citecolor=red, urlcolor=cyan, linktoc=page}
%\usepackage{cite}
\usepackage{xfrac}
\usepackage{empheq}
\usepackage{caption, subcaption}

\newcommand{\p}{\partial}
\newcommand{\oi}{\omega_i}
\newcommand{\oj}{\omega_j}
\newcommand{\ok}{\omega_k}
\newcommand{\ol}{\omega_\ell}
\newcommand{\oone}{\overline{\omega}_1}
\newcommand{\otwo}{\overline{\omega}_2}
\newcommand{\thi}{\theta_i}
\newcommand{\thj}{\theta_j}
\newcommand{\thk}{\theta_k}
\newcommand{\thl}{\theta_\ell}
\newcommand{\mc}{\mathcal}
\newcommand{\jm}{\ensuremath{j_{max}}}
\newcommand{\ob}{\overline{\omega}}
\newcommand{\oib}{\overline{\omega}_i}
\newcommand{\ojb}{\overline{\omega}_j}
\newcommand{\okb}{\overline{\omega}_k}
\newcommand{\ib}{\overline{i}}
\newcommand{\jb}{\overline{j}}
\newcommand{\kb}{\overline{k}}
\newcommand{\oal}{\omega_\alpha}
\newcommand{\obet}{\omega_{\beta}}
\newcommand{\ogam}{\omega_\gamma}

%%%%%%%%%%%%%%%%%%%%%%%%%%%%%%%%%%%%%%%%%
%%%%%%%%%%%%%%%%%%%%%%%%%%%%%%%%%%%%%%%%%

\title{Examining Instabilities Due to Driven Scalars in AdS}

\abstract{We extend the study of the non-linear perturbative theory of energy cascades in AdS to include solutions of driven systems, i.e. those with time-dependent sources on the AdS boundary. This necessitates the activation of non-normalizable modes in the massive bulk scalar field, which then couple to the metric and normalizable scalar modes. Analytic expressions for secular terms in the renormalization flow equations are determined for scalars in $AdS_{d+1}$ with any mass, and for various driving functions. We then numerically evaluate these sources for $d=4$ and discuss what role these driven solutions play in the perturbative stability of AdS.}

%%%%%%%%%%%%%%%%%%%%%%%%%%%%%%%%%%%%%%%%%
%%%%%%%%%%%%%%%%%%%%%%%%%%%%%%%%%%%%%%%%%

\begin{document}
\maketitle
\flushbottom
\newpage

%%%%%%%%%%%%%%%%%%%%%%%%%%%%%%%%%%%%%%%%%
%%%%%%%%%%%%%%%%%%%%%%%%%%%%%%%%%%%%%%%%%

\section{Introduction}

Nonlinear instabilities in Anti-de Sitter space have been the subject of examinations on mathematical, gravitational, and holographic grounds. as well as through the holographic description of quantum quenches. Numerical evolution of such systems accounting of the return to equilibrium -- signalled by the formation of a black hole in the dual theory -- requires advanced numerical methods, and has been the subject of numerous studies. While AdS has been shown to be perturbatively unstable to generic data, an interesting array of stable and meta-stable behaviours that resist gravitational collapse have been demonstrated for initial data close to the fundamental AdS modes. Data that exhibits stability over long times (a scale that is often set by the amplitude of the perturbation) exhibits inverse energy cascades that balance the direct cascade of energy to short length scales. This weakly turbulent energy cascade is captured by the third-order dynamics of a perturbative expansion. To describe the balance of inverse and direct energy flow, a second, ``slow time'' is introduced that governs the evolution of the scalar field and, therefore, the metric functions. This is known as the Two-Time Formulation (TTF), and allows for analytical determination of the evolution of the scalar field in the perturbative regime.

Conventional examinations of perturbative stability using TTF have focused on the reaction of the bulk space to some initial energy perturbation and with an aim to study the process of that . However, the analytic descriptions of holographic pumped solutions -- those with periodic boundary conditions that constantly inject energy into the system -- remains undetermined.  With this in mind, we examine the effect of a time-dependent source on the conformal boundary has on the analytic expressions for the time evolution of the slowly varying integration constants. Quenches in asymptotically AdS spacetime restrict the space of oscillon solutions to those that approach zero near the conformal boundary; however, in a driven solution the energy is pumped into the system through a second class of oscillons: those that approach constant, non-zero values on the boundary. Since these solutions will have non-finite inner products over the space, they are known as non-normalizable solutions. These non-normalizable modes couple to the metric and the normalizable modes to bring energy into the system, where direct and inverse energy cascades proceed over perturbative time scales.

Following the study of resummation and time-averaging procedures for scalar fields in AdS, we isolate secular terms -- those that grow linearly with time and cannot be absorbed by a phase term -- from those that are averaged out. The terms that persist after time averaging are those that obey certain resonance conditions between the frequencies of the non- and normalizable modes. By evaluating the third-order interactions on resonance, we use a renormalization procedure to absorb resonant contributions into the equations for the slowly varying amplitude and phase variables of the scalar field.

This paper is organized as follows: after a brief discussion of how to arrive at the third order source term, we consider the addition of a time-dependent boundary condition for the scalar field. As an exercise, and to provide explicit expressions for our choice of gauge and mass, \S\!~\ref{sec: norm res} examines the resonant contributions in the case of a massive scalar field in AdS$_{d+1}$ with any mass-squared, up to and including the Breitenlohner-Freedman mass \cite{Breitenlohner:1982bm}: $m^2_{BF} \leq m^2$. We recover the natural vanishing of two of the three resonances, and then examine the effects of mass-dependence on the non-vanishing channel. Whenever values are calculated, the choice of $d=4$ is implied as to draw the most direct comparison to existing literature. In section~\S\!~\ref{sec: NNmodes}, we extend the boundary conditions to include a variety of periodic boundary sources that couple to non-normalizable modes in the bulk. For each choice of boundary condition, we derive analytic expressions for applicable resonances and evaluate these expressions for different ranges of scalar field masses. Finally, in~\S\!~\ref{sec: discussion} we discuss the implications of non-vanishing resonances on the competing energy cascades and the perturbative stability of such systems. For completeness, we include details of our derivation of the general source term in an appendix, as well as a complete list of possible resonances and their contributions in the case of activating two, equal frequency non-normalizable modes.



%%%%%%%%%%%%%%%%%%%%%%%%%%%%%%%%%%%%%%%%%
%%%%%%%%%%%%%%%%%%%%%%%%%%%%%%%%%%%%%%%%%

\section{Source Terms and Boundary Conditions}

Let us first consider a minimally coupled, massive scalar field coupled to a spherically symmetric, asymptotically AdS$_{d+1}$ spacetime in global coordinates, whose metric is given by
\begin{align}
\label{AdS metric}
ds^2 &= \frac{L^2}{\cos(x)} \left( - A(t,x) e^{-2 \delta(t,x)} \, dt^2 + A^{-1}(t, x) \, dx^2 + \sin^2 (x) \, d\Omega^2_{d-1} \right) \, ,
\end{align}
where $L$ is the AdS curvature (hereafter set to $1$), and the radial coordinate $x \in [0, \pi/2)$. The dynamics of the system come from the Einstein and Klein-Gordon equations:
\begin{align}
G_{\mu \nu} + \Lambda g_{\mu \nu} = 8 \pi \left( \nabla_\mu \phi \nabla_\nu \phi - \frac{1}{2} g_{\mu \nu} \left( \nabla^\rho \phi \nabla_\rho \phi + m^2 \phi^2 \right) \right), \quad \text{and} \quad \nabla^2 \phi - m^2 \phi = 0 \, ,
\end{align}
with $\Lambda$ as cosmological constant for AdS, $\Lambda = -d(d-1)/2$. 

Perturbing around static AdS, the scalar field is expanded in of odd powers of epsilon 
\begin{align}
\phi(t,x) = \epsilon \phi_1 + \epsilon^3 \phi_3 + \ldots
\end{align}
and the metric functions $A$ and $\delta$ in even powers,
\begin{align}
A(t, x) = 1 + A_2 \epsilon^2 + \ldots \quad \text{and} \quad \delta(t, x) = \epsilon^2 \delta_2 + \ldots \, .
\end{align}
We choose to work in the boundary gauge, where $\delta(t, \pi/2) = 0$, for reasons that we discuss below.

At linear order, $\phi_1$ satisfies
\begin{align}
\label{phi1 eqn}
\p^2_t \phi_1 + \hat L \phi_1 = 0 \quad \text{where} \quad \hat{L} \equiv \frac{1}{\mu} (\mu' \p_x + \mu \p^2_x)  - \frac{m^2}{\cos^2(x)} \, ,
\end{align}
where $\mu \equiv \tan^{d-1}(x)$. Writing the scalar field as the product of time- and position-dependent parts,
\begin{align}
\phi_1 (t, x) = \sum_j c_j (t) e_j (x) \, ,
\end{align}
we find that the basis functions $e_j (x)$ are the solutions to the eigenvalue equation
\begin{align}
\label{eigen eqn}
\hat L e_j(x) = \omega^2_j e_j(x) .
\end{align}
The general solution to this eigenvalue equation involves two types of functions: those that vanish as $x \to \pi / 2$ and therefore are normalizable, and those that approach a finite value on the boundary and are non-normalizable. In many previous works, the dynamics of scalar fields have been studied using exclusively normalizable functions (as is the focus of \S\!~\ref{sec: norm res}); however, we now wish to consider the effects of exciting the second class of basis functions. These driven systems have a time-dependent source term on the boundary that sends energy into bulk space through coupling with non-normalizable modes. Energy is then distributed to both normalizable and non-normalizable scalar modes, as well as the metric functions. 

Writing out the full solution to \eqref{eigen eqn} involves summing over both normalizable and non-normalizable eigenmodes \cite{0712.0689},
\begin{align}
\label{phi1 gen}
\phi_1(t,x) = \sum_j c_j(t) e_j(x) + \sum_\alpha \bar A_\alpha (t) E_\alpha(x) \, .
\end{align}
The values of $\bar A_\alpha$ are set by the choice of boundary conditions. For example, if the driving term on the boundary is a single, periodic function, then
\begin{align}
\label{BC}
\phi_1(t,\pi/2) = \mc A \cos \ob t \, .
\end{align}
Extending the boundary condition to include the addition of two driving terms -- or a more general sum over Fourier modes -- would set further $\bar A_\alpha$ values. 

The normalizable modes have eigenfunctions given by
\begin{align}
e_j(x) &= k_j \left( \cos(x) \right)^{\Delta^+} P_{j}^{(d/2 - 1, \, \Delta^+ - d/2)} \left( \cos (2x) \right) \\
k_j &= 2 \sqrt{\frac{(j + \Delta^+ /2) \Gamma(j+1) \Gamma(j+\Delta^+)}{\Gamma(j+d/2) \Gamma(j + \Delta^+ - d/2 + 1)}} \, ,
\end{align} 
with fully resonant eigenvalues $\omega_j = 2j + \Delta^+$ with $j \in \mathbb{Z}^*$. We define $\Delta^+$ as the positive root of $\Delta ( \Delta - d ) = m^2$. This resonant spectrum is responsible for the secular growth of terms at third order, which must be controlled the secular term resummation procedure described below. On the other hand, the non-normalizable eigenfunctions have arbitrary frequencies $\omega_\alpha$, and are given by
\begin{align}
\label{general basis}
E_\alpha (x) =  \left( \cos(x) \right)^{\Delta_+} {_2F_1} \left(\frac{\Delta_+ + \omega_\alpha}{2}, \frac{\Delta_+ - \omega_\alpha}{2}, d/2 ; \sin^2 (x) \right) \, .
\end{align}

By allowing frequencies and basis functions to remain unspecified for the time being, we can show that the $\mc O(\epsilon^3)$ part of the scalar field satisfies the equation
\begin{align}
\label{3rd order}
\ddot \phi_3 + \hat L \phi_3 = S = 2 (A_2 - \delta_2) \ddot \phi_1 + (\dot A_2 - \dot \delta_2) \dot\phi_1 + (A_2' -\delta_2' )\phi_1' + m^2 A_2 \phi_1 \sec^2 x \, .
\end{align}
Following the steps outlined in Appendix~\ref{app: source term derivation}, we project \eqref{3rd order} onto the basis of normalizable modes and, employing the solution ${c_i(t) = a_i \cos (\oi t + b_i) = a_i \cos\theta_i}$ for the time-dependent portion of the scalar field, find that the general expression for the source term is
\begin{align}
\label{general source}
S_\ell &=\frac{1}{4} \sum_{\substack{i,j,k \\ k \neq \ell}}^\infty \frac{a_i a_j a_k \ok}{\ol^2 - \ok^2} \Big[ Z^-_{ijk\ell} (\oi + \oj - 2\ok) \cos (\thi + \thj - \thk) - Z^-_{ijk\ell} (\oi + \oj + 2\ok) \cos (\thi + \thj + \thk) - \nonumber \\
%
&\qquad + Z^+_{ijk\ell} (\oi - \oj + 2\ok)  \cos(\thi - \thj + \thk) - Z^+_{ijk\ell} (\oi - \oj - 2\ok) \cos (\thi - \thj - \thk) \Big] \nonumber \\
%
& + \frac{1}{2}\sum_{\substack{i,j,k \\ i \neq j}}^\infty a_i a_j a_k \oj \left( H_{ijk\ell} + m^2 V_{jki\ell} - 2\ok^2 X_{ijk\ell} \right) \Big[ \frac{1}{\oi - \oj} \left( \cos (\thi - \thj - \thk)  + \cos(\thi - \thj + \thk) \right) \nonumber \\
%
& \qquad - \frac{1}{\oi + \oj} \left( \cos (\thi + \thj - \thk)  + \cos ( \thi + \thj + \thk) \right) \Big] \nonumber \\
%
& - \frac{1}{4} \sum_{i,j,k}^\infty a_i a_j a_k \Big[ \left( 2\oj \ok X_{ijk\ell} + m^2 V_{ijk\ell} \right)\cos(\thi + \thj - \thk) -  \left( 2\oj\ok X_{ijk\ell} - m^2 V_{ijk\ell} \right) \cos(\thi - \thj - \thk) \nonumber \\
%
& \qquad + \left(2\oj \ok X_{ijk\ell} + m^2 V_{ijk\ell} \right) \cos (\thi - \thj + \thk) - \left( 2\oj\ok X_{ijk\ell} - m^2 V_{ijk\ell} \right) \cos(\thi + \thj + \thk) \Big] \nonumber \\
%
& + \frac{1}{4} \sum_{i,j}^\infty a_i a_j a_\ell \ol \Big[ \tilde Z^-_{ij\ell} (\oi + \oj - 2\ol) \cos (\thi + \thj - \thl) - \tilde Z^-_{ij\ell} (\oi + \oj + 2\ol) \cos(\thi + \thj +  \thl) \nonumber \\
%
& \qquad + \tilde Z^+_{ij\ell} (\oi - \oj + 2\ol) \cos(\thi - \thj + \thl)  - \tilde Z^+_{ij\ell} (\oi - \oj - 2\ol) \cos( \thi - \thj - \thl)  \Big] \nonumber \\
%
& - \frac{1}{4} \sum_{i,j}^\infty a_i^2 a_j \left( H_{iij\ell} + m^2 V_{jii\ell} - 2\oj^2 X_{iij\ell} \right) \big[ \cos (2\thi - \thj) + \cos (2\thi + \thj) \big] \nonumber \\
%
& - \frac{1}{2} \sum_{i,j}^\infty a_i^2 a_j \left( H_{iij\ell} + m^2 V_{jii\ell} - 2\oj^2 X_{iij\ell} + 4\oi^2 \oj^2 P_{j\ell i} + 2\oi^2 (M_{j\ell i} + m^2 Q_{j\ell i}) \right) \cos \thj . \hspace{-0.2in}
\end{align} 
Note that sums and restrictions on indices must be interpreted as sums and restrictions on \emph{frequencies} when any of the modes is non-normalizable, since $\omega_\alpha \neq 2 \alpha + \Delta^+$ in general.

As mentioned above, the linear growth of resonant terms with time, i.e. secular growth, at $\mc O(\epsilon^3)$ can be absorbed into the time-dependent part of the scalar field at that order \cite{hep-th/9506161}. Thus, \eqref{3rd order} tells us that
\begin{align}
\ddot c^{(3)}_\ell (t) + \ol^2 c^{(3)}_\ell(t) = S^{(3)}_\ell \cos \left( \ol t + \varphi_\ell \right) \, ,
\end{align}
where $S^{(3)}_\ell$ is a polynomial in $a_i$ determined by evaluating the resonant contributions from \eqref{general source}, and $\varphi_\ell$ is some combination of the $b_i$. To obtain the renormalization flow equations, we can rewrite the amplitudes and phases in terms of renormalized integration constants that exactly cancel the secular terms at each moment. Doing so yields the renormalization flow equations for the renormalized constants~\cite{1407.6273}
\begin{align}
\label{RN flow 1}
\frac{2 \ol}{\epsilon^2} \frac{d a_\ell}{d t} &= - S^{(3)}_\ell \sin \left( b_\ell - \varphi_\ell \right) \, , \\
\label{RN flow 2}
\frac{2 \ol a_\ell}{\epsilon^2} \frac{d b_\ell}{d t} &= - S^{(3)}_\ell \cos \left( b_\ell - \varphi_\ell \right) \, .
\end{align}
Note that the amplitudes and phases evolve with respect to the ``slow time'' $\tau = \epsilon^2 t$. In practice, once these flow equations can be written down the perturbative evolution of the system is determined up to a timescale of $t \sim \epsilon^{-2}$.

Let us now examine the exact form of $S^{(3)}_\ell$ for the case of a massive scalar field. As an exercise, we first derive the resonant contributions when the boundary source is zero, and therefore only normalizable modes are present. These results agree numerically with previous work on normalizable modes for massless scalars in the interior time gauge $(\delta(t, 0) = 0)$ \cite{1810.04753}. The definitions of the integral functions $Z$, $H$, $X$, etc. differ slightly from other works -- in part because of the gauge choice, and in part because of a desire to separate out mass-dependent terms -- and so are given explicitly in Appendix~\ref{app: source term derivation}.

%%%%%%%%%%%%%%%%%%%%%%%%%%%%%%%%%%%%%%%%%
%%%%%%%%%%%%%%%%%%%%%%%%%%%%%%%%%%%%%%%%%

\section{Resonances From Normalizable Solutions}
\label{sec: norm res}
Consider the case where each of the basis functions are given by normalizable solutions. After time-averaging, the  resonant contributions occur for the following combination of normalizable frequencies: 
\begin{align}
\label{gen res}
\oi \pm \oj \pm \ok = \pm \ol \,
\end{align}
which can be separated into the three distinct cases
\begin{align}
\oi + \oj + \ok &= \ol \qquad (+++) \\
\oi - \oj - \ok &= \ol \qquad (+--) \\
\oi + \oj - \ok &= \ol \qquad (++-) \, .
\end{align}
We will see that the first two resonances, $(+++)$ and $(+--)$, will non-trivially vanish whenever their respective resonance conditions are satisfied. This is in agreement with the results shown for the massless scalar in the interior time gauge (as they must be, since the choice of time gauge should not change the existence of resonant channels). Here we include the expressions for the naturally vanishing resonances, choosing to explicitly express the mass dependence.

%%%%%%%%%%%%%%%%%%%%%%%%%%%%%%%%%%%%%%%%%

\subsection{Naturally Vanishing Resonances: $(+++)$ and $(+--)$}
\label{ssec: zero resonance}

Resonant contributions that come from the condition $\oi + \oj + \ok = \ol$ contribute to the total source term via
\begin{align}
S_\ell = \underbrace{\sum_{i=0}^\infty \sum_{j=0}^\infty \sum_{k=0}^\infty}_{\oi + \oj + \ok = \, \ol} \Omega_{ijk\ell} \, a_i a_j a_k \cos \left( \thi + \thj + \thk \right) + \ldots \, ,
\end{align}
where the dots denote other resonances. $\Omega_{ijk\ell}$ is given by
\begin{align}
\label{omega}
\Omega_{ijk\ell} &= -\frac{1}{12}H_{ijk\ell} \frac{\oj (\oi + \ok +2\oj)}{(\oi + \oj)(\oj + \ok)} - \frac{1}{12} H_{ikj\ell} \frac{\ok (\oi + \oj + 2\ok)}{(\oi + \ok)(\oj + \ok)}- \frac{1}{12} H_{jik\ell} \frac{\oi (\oj + \ok +2\oi)}{(\oi + \oj)(\oi + \ok)} \nonumber \\
%
& \quad - \frac{m^2}{12} V_{ijk\ell} \left( 1 + \frac{\oj}{\oj + \ok} + \frac{\oi}{\oi + \ok} \right) - \frac{m^2}{12} V_{jki\ell} \left( 1 + \frac{\oj}{\oi + \oj} + \frac{\ok}{\oi + \ok} \right) \nonumber \\
%
& \quad - \frac{m^2}{12} V_{kij\ell} \left( 1 + \frac{\oi}{\oi + \oj} + \frac{\ok}{\oj + \ok} \right)  + \frac{1}{6} \oj \ok X_{ijk\ell} \left( 1 + \frac{\oj}{\oi + \ok} + \frac{\ok}{\oi + \oj} \right) \nonumber \\
%
& \quad + \frac{1}{6} \oi \ok X_{jki\ell} \left( 1 + \frac{\oi}{\oj + \ok} + \frac{\ok}{\oi + \oj} \right) + \frac{1}{6} \oi \oj X_{kij\ell} \left( 1 + \frac{\oi}{\oj + \ok} + \frac{\oj}{\oi + \ok} \right) \nonumber \\
%
& \quad - \frac{1}{12} Z^-_{ijk\ell} \left( \frac{\ok}{\oi + \oj} \right) - \frac{1}{12} Z^-_{ikj\ell} \left( \frac{\oj}{\oi + \ok} \right) - \frac{1}{12} Z^-_{jki\ell}  \left( \frac{\oi}{\oj + \ok} \right) \, .
\end{align}

The second naturally vanishing resonance comes from the condition $\oi - \oj - \ok = \ol$, and contributes to the total source term via
\begin{align}
S_\ell = \sum_{j=0}^\infty \sum_{k=0}^\infty \Gamma_{(j + k + \ell + \Delta^+) jk\ell} \, a_j a_k a_{(j+k+\ell + \Delta^+)} \cos \left( \theta_{(j+k+\ell + \Delta^+)} - \thj - \thk \right) + \ldots \, ,
\end{align}
where
\begin{align}
\label{gamma}
\Gamma_{ijk\ell} &= \frac{1}{4} H_{ijk\ell} \frac{\oj (\ok - \oi + 2\oj)}{(\oi - \oj)(\oj + \ok)} + \frac{1}{4} H_{jki\ell} \frac{\ok (\oj - \oi + 2\ok)}{(\oi - \ok)(\oj + \ok)} + \frac{1}{4} H_{kij\ell} \frac{\oi (\oj + \ok - 2\oi)}{(\oi - \oj)(\oi - \ok)} \nonumber \\
% 
& \quad -\frac{1}{2} \oj \ok X_{ijk\ell} \left( \frac{\ok}{\oi - \oj} + \frac{\oj}{\oi - \ok} - 1\right) + \frac{1}{2} \oi \ok X_{jki\ell} \left( \frac{\ok}{\oi - \oj} + \frac{\oi}{\oj + \ok} - 1 \right) \nonumber \\
%
& \quad + \frac{1}{2} \oi \oj X_{kij\ell} \left( \frac{\oj}{\oi - \ok} + \frac{\oi}{\oj + \ok} -1 \right) + \frac{m^2}{4} V_{jki\ell} \left( \frac{\oj}{\oi - \oj} + \frac{\ok}{\oi - \ok} -1\right) \nonumber \\
%
& \quad - \frac{m^2}{4} V_{kij\ell} \left( \frac{\oi}{\oi - \oj} + \frac{\ok}{\oj + \ok} + 1\right) - \frac{m^2}{4} V_{ijk\ell} \left( \frac{\oi}{\oi - \ok} + \frac{\oj}{\oj + \ok} + 1 \right) \nonumber \\
%
& \quad + \frac{1}{4} Z^-_{kji\ell} \left( \frac{\oi}{\oj + \ok}\right) - \frac{1}{4} Z^+_{ijk\ell} \left( \frac{\ok}{\oi - \oj} \right) - \frac{1}{4} Z^+_{jki\ell} \left( \frac{\oj}{\oi - \ok}\right) \, .
\end{align}

Building on the work done with massless scalars, we are able to demonstrate that \eqref{omega} and \eqref{gamma} continue to vanish for massive scalars ($m^2 \geq m^2_{BF}$) in the boundary gauge; thus, the dynamics governing the weakly turbulent transfer of energy is determined only from the remaining resonance channel. When non-normalizable modes are introduced, we will see that naturally vanishing resonances are not present and so the total third-order source term is the sum over all resonant channels.

%%%%%%%%%%%%%%%%%%%%%%%%%%%%%%%%%%%%%%%%%

\subsection{$(++-)$}
\label{subs: ttf resonances}

The first non-vanishing contributions arise when $\oi + \oj = \ok + \ol$. This contribution can be split into three coefficients that are evaluated for certain subsets of the allowed values for the indices, namely
\begin{align}
S_\ell &= T_\ell a^3_\ell \cos (\thl + \thl - \thl) + \sum_{i \neq \ell}^\infty R_{i \ell} \, a^2_i a_\ell \cos(\thi + \thl - \thi) \nonumber \\
& \qquad + \sum_{i \neq \ell}^\infty \sum_{j \neq \ell}^\infty S_{i j (i + j - \ell) \ell} \, a_i a_j a_{(i + j - \ell)} \cos(\thi + \thj - \theta_{i + j -\ell} ) \, ,
\end{align}
where each of the coefficients is given by
\begin{align}
\label{S_ppm}
S_{ijk\ell} &= - \frac{1}{4} H_{kij\ell} \frac{\oi (\oj - \ok + 2\oi)}{(\oi - \ok)(\oi + \oj)} -\frac{1}{4} H_{ijk\ell} \frac{\oj (\oi - \ok + 2\oj)}{(\oj - \ok)(\oi + \oj)} - \frac{1}{4} H_{jki\ell} \frac{\ok ( \oi + \oj - 2\ok)}{(\oi - \ok)(\oj - \ok)} \nonumber \\
%
& \quad - \frac{1}{2} \oj \ok X_{ijk\ell} \left( \frac{\oj}{\oi - \ok} - \frac{\ok}{\oi + \oj} + 1 \right) - \frac{1}{2} \oi \ok X_{jki\ell} \left( \frac{\oi}{\oj - \ok} - \frac{\ok}{\oi + \oj} + 1 \right) \nonumber \\
%
& \quad + \frac{1}{2} \oi \oj X_{kij\ell} \left( \frac{\oi}{\oj - \ok} + \frac{\oj}{\oi - \ok} + 1 \right) - \frac{m^2}{4} V_{ijk\ell} \left( \frac{\oi}{\oi - \ok} + \frac{\oj}{\oj - \ok} + 1\right) \nonumber \\
%
& \quad + \frac{m^2}{4} V_{jki\ell} \left( \frac{\ok}{\oi - \ok} - \frac{\oj}{\oi + \oj} - 1 \right) + \frac{m^2}{4} V_{kij\ell} \left( \frac{\ok}{\oj - \ok} - \frac{\oi}{\oi + \oj} - 1 \right) \nonumber \\
%
& \quad + \frac{1}{4}  Z^-_{ijk\ell} \left( \frac{\ok}{\oi + \oj}\right)  + \frac{1}{4}  Z^+_{ikj\ell} \left( \frac{\oj}{\oi - \ok}\right) + \frac{1}{4} Z^+_{jki\ell} \left( \frac{\oi}{\oj - \ok} \right) \, ,
\end{align}
\begin{align}
\label{R_ppm}
R_{i\ell} &= \left(\frac{\oi^2}{\ol^2 - \oi^2} \right) \big( Y_{i\ell \ell i} - Y_{i\ell i \ell} + \ol^2 ( X_{i\ell i \ell} - X_{\ell i \ell i}) \big) + \left(\frac{\oi^2}{\ol^2 - \oi^2}\right) \big( H_{\ell i i\ell} + m^2 V_{ii\ell \ell} - 2\oi^2 X_{\ell i i \ell} \big) \nonumber \\
%
& - \left(\frac{\ol^2}{\ol^2 - \oi^2} \right) \big( H_{i\ell i \ell} + m^2 V_{\ell i i \ell} - 2\oi^2 X_{i\ell i\ell} \big) - \frac{m^2}{4}(V_{i\ell i \ell} + V_{ii\ell \ell} ) + \oi^2 \ol^2 (P_{ii\ell} - 2P_{\ell \ell i}) \nonumber \\
%
& - \oi\ol X_{i\ell i \ell} - \frac{3m^2}{2} V_{\ell ii \ell} - \frac{1}{2} H_{ii\ell \ell} + \ol^2 B_{ii\ell} - \oi^2 M_{\ell \ell i} - m^2 \oi^2 Q_{\ell \ell i} \, ,
\end{align}
and
\begin{align}
\label{T_ppm}
T_{\ell} &= \frac{1}{2} \ol^2 \left( X_{\ell \ell \ell \ell} + 4 B_{\ell \ell \ell} -2 M_{\ell \ell \ell} - 2m^2 Q_{\ell \ell \ell} \right) -\frac{3}{4} \left( H_{\ell \ell \ell \ell} + 3m^2 V_{\ell \ell \ell \ell} \right) \, .
\end{align}

Following the form of \eqref{RN flow 1}~-~\eqref{RN flow 2}, these resonant terms set the evolution of the renormalized integration coefficients to be \cite{1412.3249}
\begin{align}
\frac{2 \ol}{\epsilon^2} \frac{d a_\ell}{d t} &= -  \sum_{i \neq \ell}^\infty \sum_{j \neq \ell}^\infty S_{i j (i + j - \ell) \ell} \, a_i a_j a_{(i + j - \ell)} \sin ( b_\ell + b_{(i+j-\ell)} - b_i - b_j ) \\
\frac{2 \ol a_\ell}{\epsilon^2} \frac{d b_\ell}{d t} &= - T_\ell a_\ell^3 - \sum^\infty_{i \neq \ell} R_{i\ell} \, a_i^2 a_\ell \nonumber \\
%
& \qquad - \sum_{i \neq \ell}^\infty \sum_{j \neq \ell}^\infty S_{i j (i + j - \ell) \ell} \, a_i a_j a_{(i + j - \ell)} \cos( b_\ell + b_{(i+j-\ell)} - b_i - b_j )
\end{align}

To examine the effects of non-zero masses on $R$, $S$, and $T$, we evaluate \eqref{S_ppm}-\eqref{T_ppm} for tachyonic, massless, and massive scalars in figure~\ref{fig: Nmodes}.

\begin{figure}
\centering
	\includegraphics[width=\textwidth]{./figures/Nmodesplot}
	\caption{Evaluating \eqref{S_ppm}-\eqref{T_ppm} over different values of $m^2$ for $\ell \leq 10$. $S_{ij(i+j-\ell)\ell}$ is denoted by filled circles connected by dash-dotted lines; $R_{i\ell}$ is denoted by filled triangles connected by solid lines; $T_{\ell}$ is denoted by large Xs connected by dotted lines. Different values of $m^2$ are denoted by the colour of each series.}
	\label{fig: Nmodes}
\end{figure}


%%%%%%%%%%%%%%%%%%%%%%%%%%%%%%%%%%%%%%%%%
%%%%%%%%%%%%%%%%%%%%%%%%%%%%%%%%%%%%%%%%%

\section{Resonances From Non-normalizable Modes}
\label{sec: NNmodes}

Now let us consider the excitation of non-normalizable modes by a driving term on the boundary of AdS. Since this boundary term is set at first order in the perturbative expansion, we must have that $\phi_3 (x \to \pi/2) = 0$; therefore, $\omega_\ell$ must correspond to a normalizable mode. What restrictions exist on the other frequencies, $\{ \oi, \oj, \ok \}$? Aside from the trivial case where all modes to be normalizable, we could imagine that one of the modes is non-normalizable. However, this would violate the boundary condition on $\phi_3$ is violated; thus, at least two modes must be non-normalizable so that the boundary condition can be satisfied. When three non-normalizable modes exist, there are two possibilities: first, that any combination of generically non-integer frequencies gives a non-integer value and does not contribute a secular term when projected onto the $\omega_\ell$ basis; second, some particular combination of the non-normalizable frequencies gives an integer frequency, in which case the procedure for determining the contribution to $S^{(3)}_\ell$ follows the same procedure as the all-normalizable case. Therefore, the pertinent question is what secular contribution to the source term results from two of $\{\oi, \oj, \oj\}$ being non-normalizable. Because this choice breaks some of the symmetries that contributed to the previous expressions for resonance channels, the resonance conditions must be re-examined starting from the source expression \eqref{general source}.

Before proceeding further, an important consideration what the effect of non-normalizable modes are on the perturbative expansion that leads to the source equations. Since non-normalizable solutions do not have well-defined norms, we do not know \emph{a priori} that the inner products described in Appendix~\ref{app: source term derivation} are still finite. To investigate this, consider the generic expression for the second-order metric function
\begin{align}
A_2 &= - \nu \int^x_0 dy \, \mu \left( (\dot \phi_1)^2 + (\phi'_1)^2 + m^2 \phi_1^2 \sec^2 x \right) \, ,
\end{align}
in the limit of $x \to \pi/2$, and let the scalar field $\phi_1$ be given by a generic superposition of normalizable and non-normalizable eigenfunctions as in \eqref{phi1 gen}. Ignoring the time-dependent contributions, we find that
\begin{align}
\lim_{\tilde x \to 0} A_2 (\tilde x \equiv \pi /2 - x) = \tilde{x}^{-\xi} \left( \frac{2 \tilde{x}^{2+d}}{2 - \xi} - \frac{\tilde{x}^d (1 + \left(\Delta^{-}\right)^2)}{\xi} \right)
\end{align}
where we have defined $\xi = \sqrt{d^2 + 4m^2}$. In the massless case, $\xi = d$ and all powers of $\tilde{x}$ are non-negative; thus, the limit is finite. For tachyonic masses of $m^2_{BF} < m^2 < 0$, $0 < \xi < d$ and the limit is again finite. However, for scalars that either saturate the Brietenlohmer-Freedman bound, or have $m^2 > 0$, part of the limit diverges. In order for the boundary to remain asymptotically AdS, counter-terms in the bulk action would be required to cancel such divergences -- a case we will not address presently. Thus, we will restrict our discussion to $m^2_{BF} < m^2 \leq 0$ to avoid these issues. A similar check on the near-boundary behaviour of $\delta_2$ shows that the gauge condition $\delta_2 (t, x=\pi/2)$ remains unchanged by the addition of non-normalizable modes given the same restrictions on the mass of the scalar field. With these restrictions in mind, let us now examine the resonances produced by the activation of non-normalizable modes.

\subsection{Two Non-normalizable Modes with Equal Frequencies}
\label{ssec: equalNN}

As a first case, let us assume that the two non-normalizable modes have equal, constant, and arbitrary frequencies, $\ob$. Applying the time-averaging procedure to the source $S_\ell$ once again eliminates all contributions except those that satisfy \eqref{gen res}. We are now free to choose any one of $\{\omega_i, \omega_j, \ok\}$ to be normalizable and consider when the resonance condition is satisfied. In particular, we find that the following combinations are resonant:
\begin{align}
\label{gen nn res 1}
\oi - \oj + \ok - \ol &= 0 \qquad \Rightarrow \qquad \text{either $\oi$ or $\ok$ is normalizable} \\
\oi + \oj - \ok - \ol &= 0 \qquad \Rightarrow \qquad \text{either $\oi$ or $\oj$ is normalizable} \\
\oi - \oj - \ok + \ol &= 0 \qquad \Rightarrow \qquad \text{either $\oj$ or $\ok$ is normalizable.}
\label{gen nn res 2}
\end{align}
When any of these resonance conditions is met, the remaining normalizable mode will have a frequency equal to $\ol$, collapsing all sums over frequencies so that
\begin{align}
\label{2genNN}
S_\ell = \overline{T}_{\ell} \, a_\ell \bar A_{\ob}^2 \cos (\thl) + \ldots \, ,
\end{align}
where the non-normalizable modes their amplitudes $\bar A_{\ob}$ set by the choice of boundary condition. Collecting the appropriate terms in \eqref{general source}, and evaluating the each possible resonance we find that
\begin{align}
\label{S:2NN}
\overline{T}_{\ell} &=  \bigg[ \, \frac{1}{2} Z^-_{\ell\ob\ob\ell} \left( \frac{\ob}{\ol + \ob} \right) + \frac{1}{2} Z^+_{\ell\ob\ob\ell} \left( \frac{\ob}{\ol - \ob} \right)  + H_{\ell \ob \ob \ell} \left( \frac{\ob^2}{\ol^2 - \ob^2} \right)  - H_{\ob\ell\ob\ell} \left(\frac{\ol^2}{\ol^2 - \ob^2} \right) \nonumber \\
%
& - m^2 V_{\ell \ob\ob\ell}  \left(\frac{\ol^2}{\ol^2 - \ob^2} \right) + m^2 V_{\ob\ob\ell\ell} \left( \frac{\ob^2}{\ol^2 - \ob^2} \right) + 2 X_{\ob\ob\ell\ell} \left( \frac{\ob^2 \ol^2}{\ol^2 - \ob^2} \right) - 2 X_{\ell\ell\ob\ob} \left( \frac{\ob^4}{\ol^2 - \ob^2} \right) \bigg]_{\ob \neq \ol} \nonumber \\
%
&  + \ol^2 X_{\ob\ob\ell\ell}  - \ob^2 X_{\ell\ell\ob\ob} - \frac{3}{2} m^2 V_{\ell\ell\ob\ob} - \frac{1}{2} m^2 V_{\ob\ob\ell\ell}  - \frac{1}{2} H_{\ob\ob\ell\ell} + \ol^2 \tilde{Z}^+_{\ob\ob\ell} - 2 \ob^2 \ol^2 P_{\ell\ell\ob} \nonumber \\
%
& - \ob^2 \left( \ol^2 P_{\ell\ell \ob} - B_{\ell\ell\ob} \right) \, .
\end{align}

Notice that the terms in the square braces only contribute when $\ob \neq \ol$. Beginning from \eqref{general source}, only terms in the square braces that are proportional to $Z^{\pm}$ are limited in this way; the remaining terms have no such restriction. However, it can be shown that integral functions with permuted indices are equal when the non-normalizable frequency equals the normalizable frequency. Upon simplification, factors of $\ol^2 - \ob^2$ are canceled and the overall contribution to $T_{\ell}$ from the terms in the braces is zero. Thus, these terms are grouped with those that have natural restrictions on the indices. 

The renormalization flow equations for two equal, constant, non-normalizable frequencies are then
\begin{align}
\frac{2 \ol}{\epsilon^2} \frac{d a_\ell}{dt} =  0 , \quad \text{and} \quad \frac{2 \ol a_\ell}{\epsilon^2} \frac{d b_\ell}{d t} = - \overline{T}_\ell a_\ell \bar A^2_{\ob} \, .
\end{align}
In figures~\ref{fig:equal_frequency_m0} and~\ref{fig:equal_frequency_m-1_0}, we evaluate \eqref{S:2NN} for $\ell < 10$ over a variety of $\ob$ values first for a massless scalar, then for a tachyonic scalar.

\begin{figure}
\centering
\includegraphics[width=\textwidth]{./figures/NN_equalfreq_sourceterms_m0_0+zoom}
\caption{{\it Left:} Evaluating \eqref{S:2NN} when $m^2 = 0$ for various choices of $\ob$. {\it Right}: The behaviour of $S_\ell$ for $\ob$ values near $\omega_0$.}
\label{fig:equal_frequency_m0}
\end{figure}

\begin{figure}[h]
\centering
\includegraphics[width=\textwidth]{./figures/NN_equalfreq_sourceterms_m-1_0+zoom}
\caption{{\it Left:} Evaluating $\overline{T}_{\ell}$ for a tachyon with $m^2 = -1.0$. {\it Right:} The behaviour of $S_\ell$ near $\omega_0 = \Delta^+ \approx  3.7$.}
\label{fig:equal_frequency_m-1_0}
\end{figure}

Other resonant contributions become possible for more restrictive values of the non-normalizable frequency, such as if $\ob$ is allowed to be an integer. These contributions are denoted by the dots in \eqref{2genNN} and are discussed briefly in Appendix~\ref{more 2NN}.

\subsection{Special Values of Non-normalizable Frequencies}

Let us now consider special values of non-normalizable frequencies that will lead to a greater number of resonance channels. While general non-normalizable frequencies do not require any such restrictions, we will find it informative to examine these special cases as they possess more symmetry in index/frequency values than the case of equal non-normalizable frequencies, but less than all-normalizable modes. 

\subsubsection{Add to an integer}
\label{ssec: add to integer}

First, we choose two of the modes to be non-normalizable with frequencies $\oone$ and $\otwo$ that add to give an integer: $\oone+ \otwo = 2n$ where $n = 1, 2, 3, \ldots$ (note that the $n = 0$ case means that both $\oone$ and $\otwo$ would need to be zero by the positive-frequency requirement and so would not contribute). Furthermore, either frequency need not be an integer and therefore the difference $|\oone - \otwo|$ will, in general, not be an integer. In \S\!~\ref{ssec: intpluschi}, we examine the case when the difference of non-normalizable frequencies is an integer.

When we consider possible resonance channels, we see that resonances can be grouped into
\begin{align}
\label{all pluses}
(++): \; \omega_i + 2n &= \omega_\ell \quad \forall \; \ell \geq n \\
(+-): \, \omega_i - 2n &=\omega_\ell \quad \forall \; n
\end{align}
for any $m^2_{BF} < m^2 < 0$. However, for a massless scalar, we have an additional channel
\begin{align}
\label{minus plus}
(-+): \, -\omega_i + 2n = \omega_\ell \quad \forall \; n \geq \ell + d
\end{align}
Adding the channels together, the total source term is
\begin{center}
{\bf Phase for non-normalizable modes?}
\end{center}
\begin{align}
\label{add to integer}
S_\ell &=   \overline{R}^{(+-)}_{(\ell + n) \ell} \, \bar A_1 \bar A_2 \, a_{(\ell + n)} \cos\left( \theta_{(\ell + n)} - 2nt \right) + \overline{T}_{\ell} \, \bar A_1 \bar A_2 \, a_\ell \cos \left( \theta_\ell \right) \nonumber \\ 
%
& \quad + \bigg[ \Theta\left( n - \ell - d \right) \overline{R}^{(-+)}_{(n - \ell - d) \ell} \ \bar A_1 \bar A_2 \, a_{(n - \ell - d)} \cos \left( \theta_{(n - \ell - d)} - 2nt \right) \bigg]_{m^2 = 0} \nonumber \\
%
& \quad + \Theta \left( \ell - n \right)  \overline{R}^{(++)}_{(\ell - n)\ell} \, \bar A_1 \bar A_2 \, a_{(\ell - n)} \cos \left( \theta_{(\ell - n)} + 2nt \right) ,
\end{align}
where the Heaviside step function $\Theta(x)$ enforces the restrictions on the indices in \eqref{all pluses} and \eqref{minus plus}. In the preceding expressions, the sum over all $\oone$, $\otwo$ such that $\oone + \otwo = 2n$ is implied, and only the restrictions on individual frequencies are included. Examining each channel in \eqref{add to integer} individually, we find
\begin{align}
\label{R1}
\overline{R}^{(++)}_{i \ell} &= - \frac{1}{4} \sum_{\otwo \neq \ol} \frac{\otwo}{\ol - \otwo} Z^{-}_{i12\ell} - \frac{1}{4} \sum_{\oone \neq \ol} \frac{\oone}{\ol - \oone} Z^{-}_{i21\ell} - \frac{1}{8n} \sum \left( \ol - 2n \right) Z^-_{12i\ell} \nonumber \\
%
& - \frac{1}{4} \sum_{\oi \neq \oone} \frac{1}{\ol - \otwo} \Big[ \oone \left( H_{i12\ell} + m^2 V_{12i\ell} - 2 \otwo^2 X_{i12\ell} \right) + (\ol - 2n) \left( H_{1i2\ell} + m^2 V_{i21\ell} - 2\otwo^2 X_{1i2\ell} \right)\Big] \nonumber \\
%
& - \frac{1}{4} \sum_{\oi \neq \otwo} \frac{1}{\ol - \oone} \Big[ \otwo \left( H_{i21\ell} + m^2 V_{21i\ell} - 2\oone^2 X_{i21\ell} \right) + (\ol - 2n) \left( H_{2i1\ell} + m^2 V_{i12\ell} - 2\oone^2 X_{2i1\ell} \right) \Big] \nonumber \\
%
& - \frac{1}{8n} \sum_{\oone \neq \otwo} \Big[ \oone H_{21i\ell} + \otwo H_{12i\ell} + m^2 \left( \oone V_{1i2\ell} + \otwo V_{2i1\ell} \right) - \left( \ol - 2n \right)^2 \left(\oone X_{21i\ell} + \otwo X_{12i\ell} \right) \Big] \nonumber \\
%
& + \frac{1}{2} \sum \Big[ \oone\otwo X_{i12\ell} + \left( \ol - 2n \right)\left( \oone X_{21i\ell} + \otwo X_{12i\ell} \right) - \frac{m^2}{2} \left( V_{i12\ell} + V_{i21\ell} + V_{12i\ell} \right) \Big]
\end{align}
The notation $X_{i12\ell}$ corresponds to evaluating $X_{ijk\ell}$ with $\omega_j = \oone$ and $\omega_k = \otwo$. Next, we find that
\begin{align}
\label{R2}
\overline{R}_{i \ell}^{(+-)} &= - \frac{1}{4} \sum \Big[ \frac{(\ol + 2n)}{2n} Z^-_{12i\ell} + 2 (\ol + 2n) \left( \oone X_{21i\ell} + \otwo X_{12i\ell} \right) \nonumber \\
%
& -\frac{\oone}{(\ol + \otwo)} \left( H_{i12\ell} + m^2 V_{12i\ell} - 2 \otwo^2 X_{i12\ell} \right) + \frac{(\ol + 2n)}{(\ol + \otwo)} \left( H_{1i2\ell} + m^2 V_{i21\ell} - 2\otwo^2 X_{1i2\ell} \right)  \nonumber \\
%
&- \frac{\otwo}{(\ol + \oone)} \left( H_{i21\ell} + m^2 V_{21i\ell} - 2\oone^2 X_{i21\ell} \right) + \frac{(\ol + 2n)}{(\ol + \oone)} \left(H_{2i1\ell} + m^2 V_{i12\ell} - 2\oone^2 X_{2i1\ell} \right)  \nonumber \\
%
&  - 2 \oone\otwo X_{i12\ell} + m^2 \left( V_{12i\ell} + V_{i12\ell} + V_{i21\ell} \right) \Big] + \frac{1}{4} \sum_{\otwo \neq \ol} \frac{\oone\otwo(\ol + 2n)}{\ol + \otwo} \left( X_{21i\ell} - X_{\ell i 12} \right) \nonumber \\
%
& + \frac{1}{4} \sum_{\oone \neq \ol} \frac{\oone\otwo(\ol + 2n)}{\ol + \oone} \left( X_{12i\ell} - X_{\ell i 12} \right).
\end{align}

When $m^2 = 0$, we have contributions from
\begin{align}
\label{R3}
\overline{R}_{i\ell}^{(-+)} &=  \frac{1}{4} \sum_{\otwo \neq \ol} \frac{\otwo}{\ol - \otwo} Z^+_{i12\ell} + \frac{1}{4} \sum_{\oone \neq \ol} \frac{\oone}{\ol - \oone} Z^+_{i21\ell} + \frac{1}{4} \sum_{i \neq \ell} \left( \frac{2n - \ol}{2n} \right) Z^-_{12i\ell} \nonumber \\
%
& \quad + \frac{1}{4} \sum_{\oone \neq \oi} \frac{1}{\oi - \oone} \Big[ \oone \left( H_{i12\ell} - 2\otwo^2 X_{i12\ell} \right) - (2n - \ol) \left( H_{1i2\ell} - 2\otwo^2 X_{1i2\ell} \right) \Big] \nonumber \\
%
& \quad + \frac{1}{4} \sum_{\otwo \neq \oi} \frac{1}{\oi - \otwo} \Big[ \otwo \left( H_{i21\ell} - 2\oone^2 X_{i21\ell} \right) - (2n - \ol) \left( H_{2i1\ell} - 2\oone^2 X_{2i1\ell} \right) \Big] \nonumber \\
%
& \quad - \frac{1}{8n} \sum_{\oone \neq \otwo} \Big[ \oone H_{21i\ell} + \otwo H_{12i\ell} - 2 \left( 2n - \ol \right)^2 \left(\oone X_{21i\ell} + \otwo X_{12i\ell} \right) \Big] \nonumber \\
%
& \quad - \frac{1}{2} \sum \Big[ (2n - \ol) \left( \oone X_{21i\ell} + \otwo X_{12i\ell} \right) - \oone \otwo X_{i12\ell} \Big] .
\end{align}
{\it NB.}\, In \eqref{R3} only, $\oi = 2i + \Delta^+ = 2i + d$ since this term requires that $m^2 = 0$ to contribute. We maintain the same notation out of convenience, despite the special case. Finally, 
\begin{align}
\label{T12}
\overline{T}_{\ell} &=  \frac{1}{2} \ol^2 \left( \tilde{Z}^+_{11\ell} + \tilde{Z}^+_{22\ell} \right)- \frac{1}{2} \Big[ H_{11\ell\ell} + H_{22\ell\ell} + m^2 \left( V_{\ell 1 1 \ell} + V_{\ell 2 2 \ell} \right) - 2 \ol^2 \left( X_{11\ell\ell} + X_{22\ell\ell} \right)  \nonumber \\
%
& \quad + 4 \ol^2 \left( \oone^2 P_{\ell \ell 1} + \otwo^2 P_{\ell \ell 2} \right) + 2\oone^2 M_{\ell \ell 1} + 2\otwo^2 M_{\ell \ell 2} + 2m^2 \left( \oone^2 Q_{\ell\ell 1} + \otwo^2 Q_{\ell \ell 2} \right) \Big] \, .
\end{align}

In figure~\ref{fig:atoi_all_m-4_0}, we compute the total source term -- modulo the amplitudes $a_i$ and $\bar A_\alpha$ -- for a tachyonic scalar with $n = 2$. Figure~\ref{fig:atoi_all_m0_0compare} provides a comparison between the value of the source term for a massless scalar between two choices of $n$: one that includes contributions from $\overline{R}_{i\ell}^{(-+)}$ and one that does not.

\begin{figure}
\centering
\includegraphics[width=\textwidth]{./figures/NNAddToInteger_source_n2_m-1_0}
\caption{{\it Left}: Source term values for a tachyonic scalar with $m^2 = -1.0$ when the frequencies of non-normalizable modes sum to $4.0$. {\it Right}: The absolute value of the sum of the source terms for each choice of $\oone$, $\otwo$.}
\label{fig:atoi_all_m-4_0}
\end{figure}

\begin{figure}[h!]
\centering
	\begin{subfigure}[b]{0.75\textwidth}
		\includegraphics[width=\textwidth]{./figures/NNAddToInteger_source_n2_m0_0}
		\label{fig:atoi_all_n2_m0}
	\end{subfigure}
	\vspace{-0.25in}
	\begin{subfigure}[b]{0.75\textwidth}
		\includegraphics[width=\textwidth]{./figures/NNAddToInteger_source_n4_m0_0}
		\label{fig:atoi_all_n4_m0}
	\end{subfigure}
	\caption{{\it Above:} The value of \eqref{add to integer} as a function of $\ell$ for a massless scalar with values of $\oone$ and $\otwo$ chosen so that $\oone + \otwo = 4$. {\it Below:} The same plot but with values chosen to satisfy $\oone + \otwo = 8$.}
	\label{fig:atoi_all_m0_0compare}
\end{figure}

\subsection{Integer Plus $\chi$}
\label{ssec: intpluschi}

Finally, let us consider the case where the non-normalizable frequencies are non-integer, but differ from integer values by a set amount. In analogue to the case where all modes are normalizable, we consider the non-normalizable frequencies to be shifted away from integer values by
\begin{align}
\label{int plus chi}
\ogam = 2\gamma + \chi \, ,
\end{align}
where $\gamma \in \mathbb{Z}^*$ (greek letters are chosen to differentiate these non-normalizable modes from normalizable modes with integer frequencies, which use roman letters). We furthermore limit $\chi$ to be non-integer\footnote{Indeed, for integer values of $\chi$, the sum or difference of two non-normalizable modes could be an integer. This would either be covered by the work in \S\!~\ref{ssec: add to integer}, or be a slight variation of it.} and set $m^2 = 0$ throughout. For this choice of non-normalizable frequencies there are no resonant contributions from the all-plus channel, unlike the naturally vanishing resonance found in \S\!~\ref{ssec: zero resonance}. Only when either $\oi + \ogam = \obet - \ol$, or $\oi + \ogam = \obet + \ol$ with $i + \gamma \geq \ell$, are resonant terms present. Let us examine each case separately.

\subsubsection{$\oi + \ogam = \obet - \ol$}
\label{sssec: intpluschi1}

When the resonance condition $\oi + \ogam = \obet - \ol$ is met, the contribution to the source term is of the form
\begin{align}
\label{intpluschi1 source}
S_\ell &= \sum_{i \neq \ell} \sum_{\gamma \neq \beta} \overline{S}^{(1)}_{i (i + \gamma + \ell) \gamma \ell} \, a_i \bar A_{(i + \gamma + \ell)} \bar A_\gamma \cos \left( \theta_i - \theta_{(i + \gamma + \ell)} + \theta_\gamma \right) \nonumber \\
%
& \qquad \qquad + \sum_\beta \overline{R}^{(1)}_{\beta \ell} \, a_\ell \bar A_\beta^2  \cos \left(\theta_\ell + \theta_\beta - \theta_\beta \right) + \ldots \, ,
\end{align}
where 
\begin{align}
\overline{S}^{(1)}_{i \beta\gamma\ell} &= \frac{1}{4} H_{\beta\gamma i \ell} \frac{ \ogam (\oi - \obet + 2\ogam)}{(\obet - \ogam)(\oi + \ogam)} - \frac{1}{4} H_{\gamma\beta i \ell} \frac{\obet(\oi + \ogam - 2\obet)}{(\oi - \obet)(\obet - \ogam)} - \frac{1}{4} H_{\gamma i \beta\ell} \frac{\oi (\ogam - \obet + 2\oi)}{(\oi - \obet)(\oi + \ogam)} \nonumber \\
%
& + \frac{1}{2} \oi \ogam X_{\beta\gamma i \ell} \left( \frac{\ogam}{\oi - \obet} - \frac{\oi}{\obet + \ogam} + 1 \right) + \frac{1}{2} \oi \obet X_{\gamma\beta i \ell} \left( \frac{\oi}{\obet - \ogam} + \frac{\obet}{\oi + \ogam} - 1 \right) \nonumber \\
%
& + \frac{1}{2} \obet \ogam X_{i\beta\gamma\ell} \left(\frac{\obet}{\oi + \ogam} - \frac{\ogam}{\oi - \obet} - 1 \right) - \frac{1}{4} Z^+_{\beta\gamma i \ell} \left( \frac{\oi}{\oi + \ol}\right)  \nonumber \\
%
& + \frac{1}{4} Z^{-}_{i\gamma\beta\ell} \left(  \frac{\obet}{\ol - \obet}  \right) + \frac{1}{4} Z^+_{i\beta\gamma\ell} \left( \frac{\ogam}{\ol + \ogam} \right) \, ,
\end{align}
and
\begin{align}
\overline{R}^{(1)}_{\beta \ell} &= \frac{1}{4} Z^{-}_{\ell \beta \beta \ell} \left( \frac{\obet}{\ol + \obet} \right) +  \frac{1}{4} Z^+_{\ell \beta \beta \ell} \left(\frac{\obet}{\ol - \obet} \right) + \frac{1}{2} H_{\ell\beta\beta\ell} \left( \frac{\obet^2}{\ol^2 - \obet^2} \right) - \frac{1}{2} H_{\beta \ell\beta\ell} \left( \frac{\ol^2}{\ol^2 - \obet^2} \right) \nonumber \\
%
& + X_{\beta\ell\beta\ell} \left( \frac{\ol^4}{\ol^2 - \obet^2} \right)  - \frac{1}{2} \obet^2  X_{\ell\beta\beta\ell} \left(\frac{\ol^2 + \obet^2 }{\ol^2 - \obet^2} \right)  - \frac{1}{2} H_{\ell\beta\beta\ell} + \ol^2 \tilde{Z}^+_{\beta\beta\ell} -2 \obet^2 \ol^2 P_{\ell\ell\beta} - \obet^2 M_{\ell\ell\beta} \, .
\end{align}

\subsubsection{$\oi + \ogam = \obet + \ol$}
\label{ssec: intpluschi2}

Similarly, when the resonance condition $\oi + \ogam = \obet + \ol$ is met, the contribution to the source term is
\begin{align}
\label{intpluschi2 source}
S_\ell &= \underbrace{\sum_{i \neq \ell} \sum_{\gamma \neq \beta}}_{i + \gamma \geq \ell} \overline{S}^{(2)}_{i (i + \gamma - \ell) \gamma \ell} \, a_i \bar A_{(i + \gamma - \ell)} \bar A_\gamma \cos \left( \theta_i - \theta_{(i + \gamma - \ell)}  + \theta_\gamma \right) \nonumber \\
%
& \qquad \qquad + \sum_\beta \overline{R}^{(2)}_{\beta\ell} \, a_\ell \bar A_\beta^2 \cos \left( \theta_\ell + \theta_\beta - \theta_\beta \right) + \ldots \, ,
\end{align}
where
\begin{align}
\overline{S}^{(2)}_{i\beta\gamma\ell} &= \frac{1}{4} H_{\beta\gamma i\ell} \frac{\ogam (\oi - \obet)}{(\obet - \ogam)(\oi - \ogam)} - \frac{1}{4} H_{\gamma\beta i \ell} \frac{\obet(\ol - \obet)}{(\obet - \ogam)(\oi - \obet)} + \frac{1}{4} H_{\beta i \gamma\ell} \frac{\oi (\ogam - \obet)}{(\oi - \obet)(\oi - \ogam)} \nonumber \\
%
& + \frac{1}{2} \oi \ogam X_{\beta\gamma i \ell} \left( \frac{\ogam}{\oi - \obet} - \frac{\oi}{\obet - \ogam} + 1 \right) + \frac{1}{2} \oi \obet X_{\gamma\beta i \ell} \left( \frac{\oi}{\obet - \ogam} - \frac{\obet}{\oi - \ogam} - 1 \right) \nonumber \\
%
& + \frac{1}{2} \obet \ogam X_{i\beta\gamma\ell} \left( \frac{\obet}{\oi - \ogam} - \frac{\ogam}{\oi - \obet} - 1 \right)  + \frac{1}{4} Z^-_{i\gamma\beta\ell} \left( \frac{\obet}{\ol + \obet}\right)  \nonumber \\
%
&
+ \frac{1}{4} Z^+_{i\beta\gamma\ell} \left( \frac{\ogam}{\ol - \ogam}\right) - \frac{1}{4}Z^+_{\beta\gamma i \ell} \left( \frac{\oi}{\oi - \ol} \right) \, ,
\end{align}
and
\begin{align}
\overline{R}^{(2)}_{\beta\ell} &= \frac{1}{4} Z^-_{\ell\beta\beta\ell} \left( \frac{\obet}{\ol + \obet} \right) + \frac{1}{4} Z^+_{\ell\beta\beta\ell} \left( \frac{\obet}{\ol - \obet} \right) + \frac{1}{2} H_{\ell\beta\beta\ell} \left( \frac{\obet^2}{\ol^2 - \obet^2} \right) - \frac{1}{2} H_{\beta\ell\beta\ell} \left( \frac{\ol^2}{\ol^2 - \obet^2} \right) \nonumber \\
%
& \!\!\!\!\!\!\!\! + X_{\beta\beta\ell\ell} \left( \frac{\ol^2}{\ol^2 - \obet^2} \right) + \frac{1}{2} \obet^2 X_{\ell\beta\beta\ell} \left( \frac{\ol^2 + \obet^2}{\ol^2 - \obet^2} \right) - \frac{1}{2} H_{\beta\beta\ell\ell} +  \ol^2 \tilde{Z}^+_{\beta\beta\ell} - 2 \obet^2 \ol^2 P_{\ell\ell\beta} - \obet^2 M_{\ell\ell\beta} \vspace{-0.3in}
\end{align}

\begin{figure}[t]
\centering
	\includegraphics[width=\textwidth]{./figures/IntPlusChi}
	\caption{{\it Left:} Evaluating the source term \eqref{intpluschi1 source} for various values of $\chi$ for $\ell < 10$. {\it Right:} Evaluating the source term \eqref{intpluschi2 source} subject to $i + \gamma \geq \ell$ for the same values of $\chi$ and the same range of $\ell$.}
	\label{fig: twoiplusx}
\end{figure}

Unlike the case with all normalizable modes where two of the three resonance channels naturally vanished, both of the resonant channels contribute when the non-normalizable modes have frequencies given by \eqref{int plus chi}. Therefore, the renormalization flow equations will contain contributions from both channels:
\begin{align}
\frac{2 \ol}{\epsilon^2} \frac{d a_\ell}{d t} &= - \sum_{i \neq \ell} \sum_{\gamma \neq \beta} \overline{S}^{(1)}_{i (i + \gamma + \ell) \gamma \ell} \, a_i \bar A_{(i + \gamma + \ell)} \bar A_\gamma \sin \left( b_\ell + b_{(i + \gamma + \ell)} - b_i - b_\gamma \right) \nonumber \\
%
& \qquad \qquad -  \underbrace{\sum_{i \neq \ell} \sum_{\gamma \neq \beta}}_{i + \gamma \geq \ell} \overline{S}^{(2)}_{i (i + \gamma - \ell) \gamma \ell} \, a_i \bar A_{(i + \gamma - \ell)} \bar A_\gamma \sin \left( b_\ell + b_{(i+\gamma - \ell)} - b_i - b_\gamma \right)
\end{align}
\begin{align}
\frac{2 \ol a_\ell}{\epsilon^2} \frac{d b_\ell}{dt} &=  - \sum_\beta \overline{R}^{(1)}_{\beta \ell} \, a_\ell \bar A_\beta^2 - \sum_\beta \overline{R}^{(2)}_{\beta\ell} \, a_\ell \bar A_\beta^2 \nonumber \\
%
& \qquad - \sum_{i \neq \ell} \sum_{\gamma \neq \beta} \overline{S}^{(1)}_{i (i + \gamma + \ell) \gamma \ell} \, a_i \bar A_{(i + \gamma + \ell)} \bar A_\gamma \cos \left( b_\ell + b_{(i + \gamma + \ell)} - b_i - b_\gamma \right) \nonumber \\
%
& \qquad \qquad -  \underbrace{\sum_{i \neq \ell} \sum_{\gamma \neq \beta}}_{i + \gamma \geq \ell} \overline{S}^{(2)}_{i (i + \gamma - \ell) \gamma \ell} \, a_i \bar A_{(i + \gamma - \ell)} \bar A_\gamma \cos \left( b_\ell + b_{(i+\gamma - \ell)} - b_i - b_\gamma \right)
\end{align}
In figure~\ref{fig: twoiplusx}, we evaluate both resonant contributions channels and plot their contributions for various values of $\chi$. In particular, we examine the values $\chi \in \{ \pi/6, \ldots, 7\pi/6 \}$.


%%%%%%%%%%%%%%%%%%%%%%%%%%%%%%%%%%%%%%%%%
%%%%%%%%%%%%%%%%%%%%%%%%%%%%%%%%%%%%%%%%%

\section{Discussion}
\label{sec: discussion}

Discussion goes here.

%%%%%%%%%%%%%%%%%%%%%%%%%%%%%%%%%%%%%%%%%
%%%%%%%%%%%%%%%%%%%%%%%%%%%%%%%%%%%%%%%%%

\acknowledgments The author would like to thank A. R. Frey for their guidance and insight with this project. 

%%%%%%%%%%%%%%%%%%%%%%%%%%%%%%%%%%%%%%%%%
%%%%%%%%%%%%%%%%%%%%%%%%%%%%%%%%%%%%%%%%%

\appendix
\section{Derivation of Source Terms For Massive Scalars}
\label{app: source term derivation}

The derivation of the general expression for the $\mc{O}(\epsilon^3)$ source term for massive scalars closely follows the massless case, particularly if one chooses not to write out the explicit mass dependence as was done in \cite{1810.04753}. However, since we have chosen to write our equations in a slightly different way -- and in a different gauge -- than previous authors, one may find it instructive to see the differences in the derivations. Below we have included the intermediate steps involved in deriving the third-order source term $S_\ell$.

Continuing the expansion of the equations of motion in powers of $\epsilon$, we see that the backreaction between the metric and the scalar field appears at second order in the perturbation,
\begin{align}
A_2' = - \mu \nu \left[ (\dot \phi_1 )^2 + (\phi_1')^2 + m^2 \phi_1^2 \sec^2 x \right] + \nu' A_2 / \nu
\end{align}
which can be directly integrated to give
\begin{align}
A_2 = -\nu \int^x_0 dy \, \mu \left( (\dot \phi_1 )^2 + (\phi_1')^2 + m^2 \phi_1^2 \sec^2 x \right) \, .
\end{align}
Similarly, the first non-trivial contribution to the lapse (in the boundary time gauge) is
\begin{align}
\delta_2 = \int^{\pi/2}_x dy \, \mu \nu \left(  (\dot \phi_1 )^2 + (\phi_1')^2 \right) \, .
\end{align}
For convenience, we have also defined the functions
\begin{align}
\mu (x) = \left( \tan x \right)^{d-1} \quad \text{and} \quad \nu(x) = (d-1) / \mu ' \, .
\end{align}

To aide in evaluating integrals and inner products, it is useful to derive several identities. First, from the equation for the scalar field's time-dependent coefficients $c_i$,
\begin{align} 
\ddot c_i + \oi^2 c_i = 0 \quad \Rightarrow \quad \p_t \left(\dot c_i^2 + \oi^2 c_i^2 \right) = \p_t \mathbb C_i = 0 \, .
\end{align}
Next, from the definition of $\hat L$,
\begin{align}
\hat L e_j = -\frac{1}{\mu} \left( \mu e'_j \right)' + m^2 \sec^2 x e_j \quad \Rightarrow \quad \left( \mu e'_j \right)' = \mu \left( m^2 \sec^2 x - \omega_j^2 \right) e_j \, .
\end{align}
By considering the expression $\left( \mu e'_i e_j \right)'$, we see that
\begin{align}
\left( \mu e'_i e_j \right) ' = \left(m^2 \sec^2 x - \oi^2 \right) \mu e_i e_j + \mu e'_i e'_j \, ,
\end{align}
which, after permuting $i, j$ and subtracting from above, gives
\begin{align}
\frac{\left[ \mu (e'_i e_j \oj^2 - e_i e'_j \oi^2 ) \right]'}{(\oj^2 - \oi^2)} = \mu m^2 \sec^2 x e_i e_j + \mu e'_i e'_j \, .
\end{align}

Projecting each of the terms in \eqref{3rd order} individually onto the eigenbasis $\{ e_\ell \}$ will involve evaluating inner products involving multiple integrals. To aide in evaluating these expressions, it is useful to derive several identities. First, from the equation for the scalar field's time-dependent coefficients $c_i$,
\begin{align} 
\ddot c_i + \oi^2 c_i = 0 \quad \Rightarrow \quad \p_t \left(\dot c_i^2 + \oi^2 c_i^2 \right) = \p_t \mathbb C_i = 0 \, .
\end{align}
Next, from the definition of $\hat L$,
\begin{align}
\hat L e_j = -\frac{1}{\mu} \left( \mu e'_j \right)' + m^2 \sec^2 x e_j \quad \Rightarrow \quad \left( \mu e'_j \right)' = \mu \left( m^2 \sec^2 x - \omega_j^2 \right) e_j \, .
\end{align}
By considering the expression $\left( \mu e'_i e_j \right)'$, we see that
\begin{align}
\left( \mu e'_i e_j \right) ' = \left(m^2 \sec^2 x - \oi^2 \right) \mu e_i e_j + \mu e'_i e'_j \, ,
\end{align}
which, after permuting $i, j$ and subtracting from above, gives
\begin{align}
\frac{\left[ \mu (e'_i e_j \oj^2 - e_i e'_j \oi^2 ) \right]'}{(\oj^2 - \oi^2)} = \mu m^2 \sec^2 x e_i e_j + \mu e'_i e'_j \, .
\end{align}

Using these identities, we evaluate each of the inner products and find that
\begin{align}
\label{inner prod 1}
\langle \delta_2 \ddot \phi_1, e_\ell \rangle &= - \sum_{i = 0}^\infty \sum_{\substack{j=0 \\ k \neq \ell}}^\infty \sum_{k=0}^\infty \frac{\ok^2 c_k}{\ol^2 - \ok^2} \left[\dot c_i \dot c_j \left(X_{k\ell ij} - X_{\ell k i j} \right) + c_i c_j \left( Y_{ij\ell k} - Y_{ijk\ell} \right) \right] \nonumber \\
& \qquad  - \sum_{i=0}^\infty \sum_{j=0}^\infty \ol^2 c_\ell \left[ \dot c_i \dot c_j P_{ij\ell} + c_i c_j B_{i j \ell} \right] \, , \\
%
\langle A_2 \ddot \phi_1, e_\ell \rangle &= 2 \sum_{i = 0}^\infty \sum_{\substack{j=0 \\ i \neq j}}^\infty \sum_{k=0}^\infty \frac{\ok^2 c_k}{\oj^2 - \oi^2} X_{ijk \ell} \left( \dot c_i \dot c_j + \oj^2 c_i c_j \right) \nonumber \\
& \qquad + \sum_{i = 0}^\infty \sum_{j = 0}^\infty \oj^2 c_j \left( \mathbb C_i P_{j \ell i} + c_i^2 X_{ii j \ell} \right) \, , \\
%
\langle \dot \delta_2 \dot \phi_1 , e_\ell \rangle &= \sum_{i = 0}^\infty \sum_{\substack{j=0 \\ k \neq \ell}}^\infty \sum_{k=0}^\infty \frac{\dot c_k}{\ol^2 - \ok^2} \left[ \p_t \left( \dot c_i \dot c_j \right) \left( X_{k\ell ij} - X_{\ell k i j} \right) + \p_t (c_i c_j) \left(Y_{ij\ell k} - Y_{ijk\ell}\right) \right] \nonumber \\
& \qquad+ \sum_{i=0}^\infty \sum_{j=0}^\infty \dot c_\ell \left[ \p_t \left( \dot c_i \dot c_j \right) P_{ij\ell} + \p_t (c_i c_j) B_{ij\ell} \right] \, , \\
%
\langle \dot A_2 \dot \phi_1, e_\ell \rangle &= -2 \sum_{i=0}^\infty \sum_{j=0}^\infty \sum_{k=0}^\infty  \dot c_k \dot c_j c_i X_{ijk\ell} \, , \\
%
\langle \left( A_2' - \delta_2' \right) \phi_1', e_\ell \rangle &= - 2 \sum_{i = 0}^\infty \sum_{\substack{j=0 \\ i \neq j}}^\infty \sum_{k=0}^\infty \frac{c_k (\dot c_i \dot c_j + \oj^2 c_i c_j)}{\oj^2 -\oi^2} H_{ijk\ell} -m^2 \sum_{i=0}^\infty \sum _{j=0}^\infty \sum_{k=0}^\infty c_i c_j c_k V_{ijk\ell} \nonumber \\
%
& \qquad - \sum_{i=0}^\infty \sum_{j=0}^\infty c_j \left[ c_i^2 H_{iij\ell} + \mathbb C_i M_{j \ell i} \right] \, , \\
%
\label{inner prod 2}
\langle A_2 \phi_1 \sec^2 x, e_\ell \rangle &= - 2\sum_{i = 0}^\infty \sum_{\substack{j=0 \\ i \neq j}}^\infty \sum_{k=0}^\infty \frac{c_k (\dot c_i \dot c_j + \oj^2 c_i c_j )}{\oj^2 - \oi^2} V_{jki\ell} \nonumber \\
& \qquad - \sum_{i=0}^\infty \sum_{j=0}^\infty c_j \left( c_i^2 V_{jii\ell} + \mathbb C_i Q_{j\ell i} \right) ,
\end{align}
where the forms of X, Y, V, H, B, M, P, and Q are given by
\begin{align}
X_{ijk\ell} &= \int^{\pi/2}_0 dx \, \mu^2 \nu e'_i e_j e_k e_\ell \\
Y_{ijk\ell} &= \int^{\pi/2}_0 dx \, \mu^2 \nu e'_i e'_j e_k e'_\ell \\
V_{ijk\ell} &= \int^{\pi/2}_0 dx \, \mu^2 \nu e_i e_j e'_k e_\ell \sec^2 x \\
H_{ijk\ell} &= \int^{\pi/2}_0 dx \, \mu^2 \nu' e'_i e_j e'_k e_\ell \\
B_{ij\ell} &= \int^{\pi/2}_0 dx \, \mu \nu e'_i e'_j \int^x_0 dy \, \mu e^2_\ell \\
M_{ij\ell} &= \int^{\pi/2}_0 dx \, \mu \nu' e'_i e_j \int^x_o dy \, \mu e_\ell^2 \\
P_{ij\ell} &= \int^{\pi/2}_0 dx \, \mu \nu e_i e_j \int^x_0 dy \, \mu e^2_\ell \\
Q_{ij\ell} &= \int^{\pi/2}_0 dx \, \mu \nu e_i e_j \sec^2 x \int^x_0 dy \, \mu e^2_\ell \, .
\end{align}
Note that, using integration by parts to remove the derivative from $\nu$ in the definitions of $H_{ijk\ell}$ and $M_{ij\ell}$, we can show that
\begin{align}
H_{ijk\ell} &= \oi^2 X_{kij\ell} + \ok^2 X_{ijk\ell} - Y_{ij\ell k}  - Y_{\ell kji}   - m^2 V_{kji\ell} -m^2 V_{ijk\ell} \, , \\
M_{ij\ell} &= \oi^2 P_{ij\ell} - B_{ij\ell} -m^2 Q_{ij\ell} \, .
\end{align}

Collecting \eqref{inner prod 1}~\!-~\!\eqref{inner prod 2} gives the expression for $S_\ell = \langle S, e_\ell \rangle$:
\begin{align}
\label{S intermediate}
S_\ell &= \sum_{\substack{i, j, k \\ k \neq \ell}}^\infty \frac{1}{\ol^2 - \ok^2} \Big[ F_k(\dot c_i \dot c_j) \left(X_{k\ell i j} - X_{\ell k i j} \right) + F_k(c_i c_j) \left(Y_{ij\ell k} - Y_{ijk\ell} \right) \Big] \nonumber \\
%
& \quad +2 \sum_{\substack{i,j,k \\ i \neq j}}^\infty \frac{c_k D_{ij}}{\oj^2 - \oi^2} \Big[  2\ok^2 X_{ijk\ell} - H_{ijk\ell} -m^2 V_{jki\ell} \Big] - \sum_{i,j,k}^\infty c_i \Big[ 2 \dot c_j \dot c_k X_{ijk\ell} + m^2 c_j c_k V_{ijk\ell} \Big] \nonumber \\ 
%
& \quad + \sum_{i,j}^\infty \Big[ F_\ell (\dot c_i \dot c_j) P_{ij\ell} + F_\ell (c_i c_j) B_{ij\ell} + 2\oj^2 c_j \left( c_i^2 X_{iij\ell} + \mathbb C_i P_{j\ell i} \right) \nonumber \\
%
& \qquad - c_j \left( c^2_i (H_{iij\ell} + m^2 V_{jii\ell} ) + \mathbb C_i (M_{j\ell i} + m^2 Q_{j\ell i}) \right) \Big] \, ,
\end{align}
where $F_k(z) = \dot c_k \dot z - 2\ok^2 c_k z$, $D_{ij} = \dot c_i \dot c_j + \omega^2_j c_i c_j$, and $\mathbb C_i = \dot c_i^2 + \oi^2 c_i^2$. Additionally, we have combined some integrals into their own expressions, namely
\begin{align}
Z^{\pm}_{ijk\ell} = \oi \oj \left( X_{k\ell ij} - X_{\ell kij} \right) \pm \left( Y_{ij\ell k} - Y_{ijk\ell} \right) \quad \text{and} \quad \tilde Z^{\pm}_{ij\ell} = \oi \oj P_{ij\ell} \pm B_{ij\ell} \, .
\end{align}
Finally, using the solution for the time-dependent coefficients, $c_i(t) = a_i(t) \cos \left( \omega_i t + b_i (t) \right) \equiv a_i \cos \theta_i$, we arrive at \eqref{general source}.

\section{Two Non-normalizable Modes with Equal Frequencies}
\label{more 2NN}
Let us return to the case of two, equal, non-normalizable modes with frequency $\ob$. Within the space of resonant frequency values, there are frequencies that happen to satisfy $\ob = \ol$ numerically and may produce extra resonances subject to restrictions on the normalizable frequency. These instances were excluded from the discussion in \S\!~\ref{ssec: equalNN}, and we address them here. When considering special integer values of $\ob$ each choice of $\ob$ below will contribute a $\overline T$-type term to the total source:
\begin{align}
\label{gen NN res 1}
\overline{T}^{(1)}_{i}: \quad \omega_i &= \ol + 2\ob \quad \forall \; \ob \in \mathbb{Z}^* \\
\overline{T}^{(2)}_{i}: \quad \omega_i &= \ol - 2\ob \quad \forall \; \ob \in \mathbb{Z}^* \; \text{such that } \ell \geq \ob \\
\label{gen NN res 2}
\overline{T}^{(3)}_{i}: \quad \omega_i &= 2\ob - \ol \quad \forall \; \ob \in \mathbb{Z}^* \; \text{such that } \ob \leq \ell + \Delta^+ \, ,
\end{align}
with $\omega_i \neq \omega_\ell$ in each case. These special values contribute to the case of two, equal non-normalizable modes via
\begin{align}
\label{2NN all}
S_\ell &= \bar A^2_{\ob} \, \overline{T}^{(1)}_{(\ell + \ob)} \, a_{(\ell + \ob)} \cos \left( \theta_{(\ell + \ob)} - 2\ob t \right) + \bar A^2_{\ob} \, \overline{T}^{(2)}_{(\ell - \ob)} \, a_{(\ell - \ob)}\cos \left( \theta_{(\ell - \ob)} + 2\ob t \right) \nonumber \\
& \quad + \bar A^2_{\ob} \, \overline{T}^{(3)}_{(\ob - \ell - \Delta^+)} \, a_{(\ob - \ell- \Delta^+)} \cos \left( 2\ob t - \theta_{(\ob - \ell - \Delta^+)} \right) 
\end{align}
under their respective conditions on the value of $\ob$. The total resonant contribution for all possible $\ob$ values is the addition of \eqref{2NN all} and \eqref{2genNN}. 
Evaluating \eqref{general source} in each case of the cases described by \eqref{gen NN res 1}~\!-~\!\eqref{gen NN res 2}, we find that
\begin{align}
\overline{T}^{(1)}_{i} &= \frac{1}{2} \bigg[ \, H_{i\ob\ob\ell} \left( \frac{\ob}{\oi - \ob} \right) - H_{\ob i \ob\ell} \left( \frac{\oi}{\oi - \ob} \right) + m^2 V_{\ob\ob i\ell} \left( \frac{\ob}{\oi - \ob} \right) \nonumber \\
%
& - m^2 V_{i \ob\ob\ell} \left( \frac{\oi}{\oi - \ob} \right) - 2 \ob^2 X_{i\ob\ob\ell} \left( \frac{\ob}{\oi - \ob} \right) + 2 \ob^2 X_{\ob i \ob\ell} \left( \frac{\oi}{\oi - \ob} \right) \bigg]_{\oi \neq \ob} \nonumber \\
%
& -\frac{1}{2} \bigg[ Z^+_{i\ob\ob\ell} \left( \frac{\ob}{\ol + \ob} \right) \bigg]_{\ol \neq \ob} \!\! + \frac{1}{4} Z^-_{\ob\ob i \ell} \left( \frac{\ol + 2\ob}{2 \ob} \right) + \frac{1}{2} \ob^2 X_{i\ob\ob\ell} - \frac{m^2}{4} V_{\ob\ob i \ell} \nonumber \\
%
& - \ob \oi X_{\ob\ob i\ell} - \frac{m^2}{2} V_{i \ob\ob\ell} \, ,
\end{align}
\begin{align}
\overline{T}^{(2)}_{i} &=  - \frac{1}{2} \bigg[ \, H_{i\ob\ob \ell} \left( \frac{\ob}{\oi + \ob} \right) + H_{\ob i \ob \ell} \left( \frac{\oi}{\oi + \ob} \right) + m^2 V_{\ob \ob i \ell} \left( \frac{\ob}{\oi + \ob} \right) \nonumber \\
%
& + m^2 V_{i\ob\ob\ell} \left( \frac{\oi}{\oi + \ob} \right) - 2 \ob^2 X_{i \ob\ob\ell} \left( \frac{\ob}{\oi + \ob} \right) - 2 \ob^2 X_{\ob\ob i\ell} \left( \frac{\oi}{\oi + \ob} \right) \bigg]_{\oi \neq \ob} \nonumber \\
%
& - \frac{1}{2} \bigg[ \, Z^{-}_{i \ob \ob \ell} \left( \frac{\ob}{\ol - \ob} \right) \bigg]_{\ol \neq \ob} \!\!\!\! - \frac{1}{4} Z^-_{\ob\ob i \ell} \left( \frac{\ol - 2\ob}{\ob} \right) + \frac{1}{2} \ob^2 X_{i\ob\ob\ell} + \frac{m^2}{4} V_{\ob\ob i\ell} \nonumber \\
%
& + \ob \oi X_{\ob\ob i \ell} + \frac{m^2}{2} V_{i \ob\ob \ell} \, ,
\end{align}
and
\begin{align}
\overline{T}^{(3)}_{i} &= \frac{1}{2} \bigg[ \, H_{i\ob\ob\ell} \left( \frac{\ob}{\oi - \ob} \right) - H_{\ob i \ob\ell} \left( \frac{\oi}{\oi - \ob} \right) + m^2 V_{\ob\ob i \ell} \left( \frac{\ob}{\oi - \ob} \right) \nonumber \\
%
& - m^2 V_{i\ob\ob\ell} \left( \frac{\oi}{\oi - \ob} \right) - 2 \ob^2 X_{i\ob\ob\ell} \left( \frac{\ob}{\oi-\ob} \right) + 2 \oi^2 X_{\ob\ob i\ell} \left( \frac{\ob}{\oi-\ob} \right) \nonumber \\
%
& - Z^+_{i\ob\ob\ell} \left( \frac{\ob}{\oi - \ob} \right) \bigg]_{\oi \neq \ob} + \frac{1}{4} Z^-_{\ob\ob i \ell} \left( \frac{2\ob - \ol}{2\ob} \right) + \frac{1}{2} \ob^2 X_{i\ob\ob\ell} - \frac{m^2}{4} V_{\ob\ob i\ell} \nonumber \\
%
&  - \ob \oi X_{\ob\ob i\ell} - \frac{m^2}{2} V_{i\ob\ob\ell} \, .
\end{align}


%%%%%%%%%%%%%%%%%%%%%%%%%%%%%%%%%%%%%%%%%
%%%%%%%%%%%%%%%%%%%%%%%%%%%%%%%%%%%%%%%%%

\bibliographystyle{JHEP}
\bibliography{DrivenTTF}

%%%%%%%%%%%%%%%%%%%%%%%%%%%%%%%%%%%%%%%%%
%%%%%%%%%%%%%%%%%%%%%%%%%%%%%%%%%%%%%%%%%

\end{document}

%%%%%%%%%%%%%%%%%%%%%%%%%%%%%%%%%%%%%%%%%
%%%%%%%%%%%%%%%%%%%%%%%%%%%%%%%%%%%%%%%%%
